\documentclass[12pt]{article}
\renewcommand{\baselinestretch}{1.2}

%revision by CZ---abstract not done
%\usepackage{latex8}
%\usepackage{times}
\usepackage{latexsym}
\usepackage{graphicx}
\usepackage{algorithm}
\usepackage{algorithmic}
%\pagestyle{empty}

%-------------------- HYPHENATIONS ------------------------
\hyphenation{PAGED-ARRAY LINKED-LIST}

%-------------------- DEFINITIONS ------------------------
\newcommand{\phsp}{\hspace*{1cm}}
\newcommand{\nhsp}{\hspace*{-1cm}}
\newcommand{\inv}{\vspace*{-0.2cm}}

\newcommand{\LDL}{\mbox{$\cal LDL$}}
\newcommand{\gl}{\gamma_{L}}
%---------------------------------------------------------
\newcommand{\bldl}{\smallskip\[\begin{array}{ll}\small}
\newcommand{\cldl}{\vspace*{-0.5cm}\[\begin{array}{ll}\small}
\newcommand{\eldl}{\end{array}\]\rm}
\newcommand{\prule}[2]{\tt #1 \leftarrow & \tt #2 \\}
\newcommand{\lrule}[2]{\tt #1 \leftarrow & \tt #2 \ \ \ \ \ \ \Box\\}
\newcommand{\pfact}[2]{\tt #1 & \tt #2 \\}
\newcommand{\pbody}[2]{\tt #1 & \tt #2 \\}
\newcommand{\lbody}[2]{\tt #1 & \tt #2 \ \ \ \ \ \Box\\}
%---------------------------------------------------------

%%%%%%%%%%%%%%%%%%%from KDD%%%%%%%%%%%%%%%%%%%%%%%%%%%%
\newtheorem{theorem}{Theorem}
\newtheorem{lemma}[theorem]{Lemma}
\newtheorem{claim}[theorem]{Claim}
\newtheorem{proposition}[theorem]{Proposition}
\newtheorem{corollary}[theorem]{Corollary}
\newtheorem{example}{Example}
\newtheorem{definition}{Definition}

\def\newdef#1{\emph{#1}}
%SQL displays
%\def\cdf{\sffamily}
\def\cdf{\sf }
\def\cdf{\sf }
%%%%%%%%%%%%%%%%%%%%%%%%%%%
\newcommand{\AXL}{$\cal AXL$}
\newcommand{\RTREE}{{\small{\cdf RTree}}}
\newcommand{\RTreeCursor}{{\small{\cdf RTreeCursor}}}
\newcommand{\DBNOTFOUND}{{\small{\cdf DB\_NOTFOUND}}}
\newcommand{\DBNOOVERWRITE}{{\small{\cdf DB\_NOOVERWRITE}}}
\newcommand{\DBKEYEMPTY}{{\small{\cdf DB\_KEYEMPTY}}}
\newcommand{\DBGETBOTH}{{\small{\cdf DB\_GET\_BOTH}}}
\newcommand{\DBSETRECNO}{{\small{\cdf DB\_SET\_RECNO}}}
\newcommand{\DBGETOIDOFKEY}{{\small{\cdf DB\_GET\_OIDOFKEY}}}
\newcommand{\DBGETOID}{{\small{\cdf DB\_GET\_OID}}}
\newcommand{\DBFIRST}{{\small{\cdf DB\_FIRST}}}
\newcommand{\DBNEXT}{{\small{\cdf DB\_NEXT}}}
\newcommand{\DBLAST}{{\small{\cdf DB\_LAST}}}
\newcommand{\DBPREV}{{\small{\cdf DB\_PREV}}}
\newcommand{\DBCURRENT}{{\small{\cdf DB\_CURRENT}}}
\newcommand{\DBNEXTDUP}{{\small{\cdf DB\_NEXT\_DUP}}}
\newcommand{\DBSET}{{\small{\cdf DB\_SET}}}
\newcommand{\DBAPPEND}{{\small{\cdf DB\_APPEND}}}
\newcommand{\dbrecnot}{{\small{\cdf db\_recno\_t}}}

%%%%%%%%%%%%%%%%%%%%%%%%%%%
\def\brack#1{\langle{#1}\rangle}
\def\newdef#1{\emph{#1}}
\def\codefont{\sffamily} %Note: < and > are not in this font; use $<$ and $>$
\def\inv{\vspace*{-0.16cm}}
\def\smb{\hspace*{-0.16cm}}
\def\kw#1{\bf{\cdf #1}}
\def\mysize#1{\normalsize{#1}}
\newenvironment{codedisplay}
{\renewcommand{\baselinestretch}{1}\vspace*{-\partopsep}\cdf\addtolength{\baselineskip}{-1pt}
\samepage\begin{tabbing}
\quad
%\ \ \ \ \ \ \ \ \=\ \ \ \ \ \ \ \ \=\ \ \ \ \ \ \ \ \=\ \ \ \ \ \ \ \ \=
%\ \ \ \ \ \ \ \ \=\ \ \ \ \ \ \ \ \=\ \ \ \ \ \ \ \ \=\ \ \ \ \ \ \ \ \=
\ \ \=\ \ \ \ \=\ \ \ \ \=\ \ \ \ \=
\ \ \=\ \ \ \ \=\ \ \ \ \=\ \ \ \ \=
\ \ \=\ \ \ \ \=\ \ \ \ \=\ \ \ \ \=
\kill}{\end{tabbing}\vspace*{-\partopsep}%\vspace*{-\topsep}\vspace*{-\parsep}
}

\newenvironment{codeexample}[1]{\begin{example}{#1}\begin{codedisplay}}{\end{codedisplay}\end{example}}

%\markright{Paper id 249}
\pagestyle{plain}

\addtolength{\topmargin}{-0in}
\addtolength{\textheight}{+0.95in}
\addtolength{\oddsidemargin}{-0.5in}
\addtolength{\textwidth}{+1in}

%%%%%%%%%%%%%%%%%%%%%%%%%%%%%%
\begin{document}

%\thispagestyle{empty}

%\newpage
\setcounter{page}{1}

\begin{center}
{\LARGE \bf R-Tree Database Design and API}\\
Version 1.0\\
%\vspace{0.25cm}
%{\large {\bf Jiejun Kong}, {\bf Haixun Wang} and {\bf Carlo Zaniolo}\\
%Computer Science Department\\ 
%University of California at Los Angeles\\ 
%Los Angeles, CA 90095}\\
%\small{ \{jkong,hxwang,zaniolo\}@cs.ucla.edu}
%\maketitle
\end{center}


%==================================================

%\begin{abstract}
%\end{abstract}

%\tableofcontents

%==================================================
\section{Design}

The R-Tree database for {\AXL} is designed to implement,
\begin{enumerate}
\item Antonin Guttman's design of R-Tree index;
\item Sequential scan over all index entries and data records;
\item Insertion/Deletion/Update of index entries and data records;
\item Finally, a BerkeleyDB compatible interface to support AXL
      queries.
\end{enumerate}

This RTree implementation is based on the original spatial data
structure described in the paper by Antonin Guttman ``R-trees: A
Dynamic Index Structure for Spatial Searching''. This document
describes the data structures involved in implementing the
R-Tree.  There are two existing online implementations.  One from
Dimitris Papadias and Yannis Theodoridis's research group (named
as the Greek implementation later in this document), and another
from General Electric company (named as the GE implementation
later in this document).  Both implementations have some
shortcoming for developing a BerkeleyDB compatible interface for
AXL.
\begin{itemize}
\item The Greek implementation is an academic experiment.  It is
      an implementation with R-Tree index and records residing on
      secondary storage.  However, there is no formal document
      describing how the secondary storage is organized (at least
      no comparable document as we will show below in this
      document), and the code has less comments than enough.
\item The GE implementation follows Guttman's specification and
      has enough code comments.  However, it is an implementation
      in primary storage only, which is unacceptable to this
      project.
\end{itemize}
Moreover, the most important reason is that neither of the above
implementations support sequential scan over all leaf nodes in the
R-Tree index.  As it is almost impossible to modify the above
implementations to satisfy our project's requirements, I have to
implement a new R-Tree structure which is a {\bf superset of
Antonin Guttman's specification-----that is, I chain all leaf
nodes together and hence allow a sequential scan over all the
records.}

I have included both implementations in the R-Tree archive.  One
is under the subdirectory ``Greek'' and the other is under the
subdirectory ``GE''.  Users may compare them with the new
implementation described in this document.

In order to fulfill specifications to allow for spatial
data storage on the R-Tree, we needed to make sure we had an
efficient data structure that allowed for operational
functionality, including insertion, deletion, searching.

Like DB2/Oracle, each relation in our R-Tree database consists of
two disk files, one for the data record, and one for the R-Tree
index.   Both files (index.db and data.db) begin with a
super-structure.
\begin{itemize}
\item In index.db we have a super block that specifies the number
      of free blocks, the last block in use (i.e., end of file),
      R-Tree parameter $m$ (i.e., minimal entries allowed in each
      R-Tree node), R-Tree parameter $M$ (i.e., maximal entries
      allowed in each R-Tree node), and the head pointer of the
      leaf node chain.

      The block length is easily adjustable. A larger block length
      allows for more nodes to be pointed to per level. In our
      sample case, we have a block length of 1024.
\item In data.db we have a supertuple that specifies the number of
      free tuples, the last tuple in use.
\end{itemize}

In index.db, the root node always resides in the block follows the
superblock (if count from zero, superblock=0, root=1).  In each
node, the basic entry we use (represented by $dp$ in the diagrams)
consists of 5 long integers (20 bytes).  Four long integers are
for the entry's rectangle coordinates (upper left corner
$(x_{ul},y_{ul})$ and lower right corner $(x_{lr},y_{lr})$), and
one long integer for the offset pointer (for non-leaf nodes this
pointer points to another node inside index.db, or for leaf nodes
this pointer points to a tuple inside data.db).

Each node in index.db also has an overhead entry holding:
\begin{enumerate}
\item (1 byte) type of the node: 'l' for leaf node, 'n' for
      non-leaf node.
\item (4 bytes) number of entries currently in use. It is a
      unsigned integer.
\item (2*4=8 bytes) if current node is a leaf node (i.e., node
      type is 'l'), then there are two pointers $prevlptr$ and
      $nextlptr$ pointing to the previous node and next node in
      the leaf chain.  The value include the pointers could be 0
      which means end of chain (block 0 is the superblock,
      definitely not a leaf node).
\end{enumerate}
Therefore, for a 1024 block size, we can have 50 entries per block
(13 bytes for the overhead and 50*20 bytes for the
entries).  In this paper we define $M$ as the maximal and $m$ as
the minimal number of entries in a node.

\paragraph{Free block management} is achieved by maintaining a
free block chain.  For index.db, the super block's $fb$ points to
the head of the chain (i.e., the first free block).  Each free
block's first 4-byte is a pointer to the next free block. If the
pointer is 0, then the end of the chain is met.  It is simple to
put a new block $X$ into the chain, we put the current head of the
chain in $X$ and record $X$ as the new head of the chain.


\paragraph{Searching [Guttman: section 3.1]}
To search, we specify a rectangular coordinate $S$, starting at
the root node $T$ of the given R-Tree index.  We return all
entries that contain those coordinates within $T$. If $T$ is not a
leaf, we check each entry $E$ to see if $E$ overlaps $S$. For all
containing entries, we recursively search upon the tree whose root
node is pointed to by $E$. If $T$ is a leaf we check all entries
$E$ to see if they match.  If they do, $E$ is a qualifying record
and we return it.

\paragraph{Insertion [Guttman: section 3.2]}
Insertion is similar to the common BTree implementation where new
index records are added to leaves and overflowed nodes are split
resulting in node propagation up the tree. To insert a node $E$,
first invoke {\em ChooseLeaf} which descends from the node until
it finds the leaf with the least enlargement result, and set $N$
to be the root node. If it is a leaf node and isn't full, we
install $E$, otherwise we invoke {\em SplitNode} to get two new
nodes $L$ and $LL$ both containing $E$ and the $M$ entries of the
old node.  We propagate the changes by calling {\em AdjustTree},
which ascends from a leaf node $L$ to the root, adjusting covering
rectangles and propagating node splits as necessary. If the node
propagation caused the root to split, we make a new root where the
children are the two resulting nodes.

\paragraph{Deletion [Guttman: section 3.3]}
In deletion we specify index record $E$ and root node $T$ and
invoke {\em FindLeaf} to locate the leaf node $L$ containing
$E$. If $T$ isn't a leaf node, it checks to see if each entry $F$
in $T$ contains the specified $E$. For each entry, we recursively
call {\em FindLeaf} to parse the entries to find the node that
points to $E$, or until all containing entries have been
checked. For each $T$ that is a leaf, we check for entries
matching with $E$. If $E$ is found we return T. We stop if the
record was not found.  Otherwise, we remove $E$ from $L/T$.
Afterwards we propagate the changes up with {\em CondenseTree},
passing $L$.  {\em CondenseTree} will eliminate the node $L$ if it
has too few entries and propagate the node elimination upwards as
needed.  All entries in the eliminated nodes are recorded in a
temporary buffer $Q$. If the root node has only one child after
the tree has been adjusted make the child the new root.  Finally
all the eliminated entries in buffer $Q$ are re-inserted into the
index again (This is the standard behavior specified by Antonin
Guttman).



\paragraph{Node Splits [Guttman: section 3.5]}
In the scenario where we have to split a node because either we
have $N$ (number of entries) $< m$ or $N > M$. To split the tree
we have three options: an exhaustive algorithm, a quadratic cost
algorithm, and a linear-cost algorithm. The exhaustive algorithm
is the most straightforward way, but not surprisingly unfeasible
for databases with many nodes. The number of possibilities is
approximately $2^M-1$.

The quadratic-cost algorithm attempts to find a small-area split
for the database. It is efficient, effective, but not complete in
search space (not guaranteed to find the smallest area
possible). It works by picking two of $M+1$ entries that would
waste the most space if paired together. For each step, we call
PickNext and select the next entry that has the largest difference
between that new entry and the groups; $ABS(MAX(d1-d2))$ where
$d1$ is the distance between the object and the first group and
$d2$ the same for the second group.

We use quadratic-cost algorithm in the implementation.

\newpage
\paragraph{Graphical Representation of the databases:}

Here is a graphical representation of index.db.
\renewcommand{\baselinestretch}{0.8}
\begin{verbatim}
+----------------------+
|fb1|tb|m|M|l1|        | superblock
+----------------------+
|ty|ni||dp|dp|.|.|.|   | root
+----------------------+
|fb2|                  | fb1 (free block 1)
+----------------------+
|ty|ni|0 |l2||dp|dp|   | l1(leaf1)
+----------------------+
|ty|ni|l1|l3||dp|dp|   | l2(leaf2)
+----------------------+
|fb3|                  | fb2 (free block 2)
+----------------------+
|          .           |
|          .           |
|          .           |
+----------------------+
|0|                    | fbN (free block N)
+----------------------+
|ty|ni|lN-1|0 ||dp|dp| | lN(leafN)
+----------------------+
tb

\end{verbatim}
\renewcommand{\baselinestretch}{1.2}

\renewcommand{\baselinestretch}{0.8}
\begin{verbatim}
size in bytes of each data segment for each block.
+------------------------------+
|1|4|4,4|20|.|.|.|......    |20|
+------------------------------+    
- Under superblock
  * fb means head of free block list. It points to the first
    free block.
  * tb means total number of blocks in current file, i.e., the block
    after the last block in used.
  * m means minimum number of entries per node
  * M means maximum number of entries per node
  * lptr means the head of the leaf node chain.


+---------------------------------+
|ty|ni|nextlptr|prevlptr|dp|dp|...| 
+---------------------------------+
The second block in index.db is always the root block.
- Under root and other nodes,
  * ty means node type, i.e., if it is a leaf or non-leaf node.
  * ni means number of entries in use in the node.
  * nextlptr/prevlptr mean for a leaf node, next (previous) leaf block POINTER 
	(i.e block number * BLOCK_SIZE).
	For the first leaf node, prevlptr=0;
	For the case that there is only one leaf, i.e. there is only 2 nodes, the
	root and the leaf in the RTREE, prevlptr=nextlptr=0; 
  * dp means data structure or leaf/non-leaf node pointer. We can
    have multiple pointers per row which represent the nodes on
    the R-tree. If the  block you are currently on is a non-leaf
    node block, then dp will point to another block (by Pointer which is Block No.
    * BLOCK_SIZE)
    in index.db. If the block you are currently on is a leaf node
    block, then dp will be a logical record number.  Currently, we
    use Berkeley DB's RECNO (variable-length record) as the underlying 
    data storage method.  Other methods, such as BTree, Queue (fixed-
    length record) in Berkeley DB may be applicable.  BDB Queue might be more
    efficient but I couldn't make it work.

\end{verbatim}
\renewcommand{\baselinestretch}{1.2}

% the following data.db design is obselete, comment it!
\newcommand{\obselet}{
\newpage
Here is a graphical representation of data.db.
\renewcommand{\baselinestretch}{0.8}
\begin{verbatim}
+--+--+--------------+
|fb|tb|	             | supertuple
+--+--+--------------+
|data                | data tuple1
+--------------------+
|data                | data tuple2
+--------------------+
|                    |
+--------------------+
|                    |
+--------------------+
|                    |
+--------------------+
|        .           |
|        .           |
|        .           |
|        .           |
+--------------------+
|                    | data tuple
+--------------------+

- Under supertuple
  * fb means head of free block list. It points to the next free block.
  * tb means total number of blocks in current file

- Under a free tuple
  * The first integer is a pointer pointing to next free tuple in
    the free-chain.  If the pointer is 0, it means the end of the
    chain.  Then the next free tuple will be grabbed from the end
    of the storage file which is recorded in tb.
\end{verbatim}
}

\renewcommand{\baselinestretch}{1.2}



\section{Implementation}
\begin{itemize}
\item {\tt rtree.c} implements R-Tree index.
\item {\tt block.c} implements node-level operations in the R-Tree
      index.
\item {\tt rectangle.c} implements operations related to
      rectangles.
\item {\tt interface.c} implements BerkeleyDB compatible
      operations including cursor operations.
\item {\tt tupledb.c} implements data record storage.
\item {\tt tupledb.h} specifies definitions and macros for data
      record storage.
\item {\tt rtree.h} specifies definitions and macros for R-Tree
      index storage.
\end{itemize}



\newpage
%==================================================
\section{API: Application Programming Interface}

\subsection{rtree\_create}
\begin{verbatim}
#include <rtree.h>
int rtree_create(RTree **rtree, char *indexFileName, char *recordFileName, int flags);
\end{verbatim}
\begin{description}
\item[Description]\ \\
  The {\em rtree\_create} function creates handle for an R-Tree
  relation.  A pointer to this structure is returned in the memory
  referenced by AXL.

  The currently supported R-Tree index entry format consists of
  two parts: (i) 4 long integers representing the coordinates of
  upperleft corner $(x_{ul},y_{ul})$ and lowerright corner
  $(x_{lr},y_{lr})$. (ii) 1 long integer representing the file offset
  to the child node.  The index is stored in {\em indexFileName}.
  The details format of the file is explain in design section of
  this document.

  The currently supported R-Tree record format is fixed length
  tuples stored in {\em recordFileName}.  The details format of
  the file is explain in design section of this document.
  
  The flags value must be set to 0.

\item[Return Value]\ \\
  The {\em rtree\_create} function returns a non-zero error value on
  failure and 0 on success.
\item[Errors]\ \\
  The {\em rtree\_create} function may fail and return a non-zero
  value -1 for disk access errors and memory exhaustion errors.
\item[Sample Call]\ 
\begin{verbatim}
RTree *rtree;

if((ret = rtree_create(&rtree, "index.db", "data.db", 0) != 0)
    return ERROR;
\end{verbatim}
\end{description}

\newpage
\subsection{{\RTREE}-$>$open}
\begin{verbatim}
#include <rtree.h>
int rtree->open(RTree **rtree, char *indexFileName, char *recordFileName, int flags);
\end{verbatim}
\begin{description}
\item[Description]\ \\
  The {\RTREE}-$>$open interface opens the database represented by name for
  both reading and writing.

  The {\em indexFileName} and {\em recordFileName} must already
  been created by the {\em rtree\_create} routine.
  {\RTREE}-$>$open is merely for re-open the existing database.

  The flags parameter must be set to 0 currently.  

\item[Return Value]\ \\
  The {\RTREE}-$>$open function returns a non-zero error value on failure
  and 0 on success.
\item[Errors]\ \\
  The {\RTREE}-$>$open function may fail and return a non-zero
  value -1 for disk access errors and memory exhaustion errors.
\item[Sample Call]\ 
\begin{verbatim}
RTree *rtree;

if((ret = rt->open(&rt, "index.db", "data.db", 0)) != 0)
    return ERROR;
\end{verbatim}
\end{description}

\newpage
\subsection{{\RTREE}-$>$close}
\begin{verbatim}
#include <rtree.h>
int RTree->close(RTree *rtree, int flags);
\end{verbatim}
\begin{description}
\item[Description]\ \\
The {\RTREE}-$>$close function closes any open cursors, frees any
allocated resources.

The flags parameter must be set to 0 currently.  

Once {\RTREE}-$>$close has been called, regardless of its return, the
{\RTREE} handle may not be accessed again until the handle is
re-opened.
\item[Return Value]\ \\
The {\RTREE}-$>$close function returns a non-zero error value on failure
and 0 on success.
\item[Errors]\ \\
  The {\RTREE}-$>$close function may fail and return a non-zero
  value -1 for disk access errors and memory exhaustion errors.
\item[Sample Call]\ 
\begin{verbatim}
if((ret = rt->close(rt, 0)) != 0)
    return ERROR;
\end{verbatim}
\end{description}

\newpage
\subsection{{\RTREE}-$>$del}
\begin{verbatim}
#include <rtree.h>
int RTree->del(RTree *rel, Rectangle *mbr, int flags);
\end{verbatim}
\begin{description}
\item[Description]\ \\
  The {\RTREE}-$>$del function removes the key/data pair from a
  relation. The key/data pair associated with the specified {\em
  mbr} is discarded from the relation. In the presence of
  duplicate key values, only the first records associated with the
  designated key will be discarded.

  The flags parameter is currently unused, and must be set to 0.
\item[Return Value]\ \\
  The {\RTREE}-$>$del function returns a non-zero error value on
  failure, 0 on success, and returns {\DBNOTFOUND} if the specified key
  did not exist in the file.
\item[Errors]\ \\
  The {\RTREE}-$>$del function may fail and return a non-zero
  value -1 for disk access errors and memory exhaustion errors.
\item[Sample Call]\ 
\begin{verbatim}
if((ret = rt->del(rt, &rectangle, 0)) != 0)
    return ERROR;
\end{verbatim}
\end{description}

\newpage
\subsection{{\RTREE}-$>$get}
\begin{verbatim}
#include <rtree.h>
int RTree->get(RTree *rtree, Rectangle *mbr, char *tuple, size_t tplsz, int flags);
\end{verbatim}
\begin{description}
\item[Description]\ \\
  The {\RTREE}-$>$get function retrieves key/data pair from the
  database.  The data record associated with the rectangle {\em
  mbr} is returned in the placeholder {\em tuple}.   The caller
  must pre-allocate the placeholder with at least size {\em
  tplsz}, as the routine simply memcpy {\em tplsz} bytes into the
  placeholder from data record storage.

  In the presence of duplicate key values, {\RTREE}-$>$get will return the
  first data item for the designated key.  Retrieval of duplicates
  requires the use of cursor operations. See
  {\RTreeCursor}-$>$c\_get for details.

  The flags parameter must be set to 0.
  
\item[Return Value]\ \\
  If the requested key is not in the database, the {\RTREE}-$>$get
  function returns {\DBNOTFOUND}.

  Otherwise, the {\RTREE}-$>$get function returns a non-zero error
  value on failure and 0 on success. 

\item[Errors]\ \\
  The {\RTREE}-$>$get function may fail and return a non-zero error
  for errors specified for other Berkeley DB and C library or system
  functions.

\item[Sample Call]\ 
\begin{verbatim}
RTree *rtree;
Rectangle mbr;
size_t tplsz = 6;
char *tuple=malloc(tplsz);

......

if((ret = rtree->get(rtree, &mbr, tuple, tplsz, 0)) != 0)
    return ERROR;

......
free(tuple);
\end{verbatim}
\end{description}

\newpage
\subsection{{\RTREE}-$>$put}
\begin{verbatim}
#include <rtree.h>
int RTree->put(RTree *rtree, Rectangle *mbr, char *tuple, size_t tplsz, int flags);
\end{verbatim}
\begin{description}
\item[Description]\ \\
  The {\RTREE}-$>$put function stores key/data pairs into the
  R-Tree index and data record.

  The flags parameter must be set to 0 or one of the following values: 
  \begin{description}
  \item[{\DBAPPEND}]\ \\
    Insert the tuple to the existing database, no matther whether
    the key {\em mbr} already exists in the database.
  \item[{\DBNOOVERWRITE}]\ \\
    Insert the tuple to the existing database, or do nothing if
    the key {\em mbr} already exists in the database.
  \end{description}

  The default behavior of the {\RTREE}-$>$put function is to enter the
  new key/data pair, replacing the first previously existing key
  {\em mbr}.
\item[Return Value]\ \\
  The {\RTREE}-$>$put function returns a non-zero error value on
  failure, 0 on success.
\item[Errors]\ \\
  The {\RTREE}-$>$put function may fail and return a non-zero
  value -1 for disk access errors and memory exhaustion errors.
\item[Sample Call]\ 
\begin{verbatim}
RTree *rtree;
Rectangle mbr;
char tuple[6]="hello";

......

if((ret = rtree->put(rtree, &mbr, tuple, 6, 0)) != 0)
    return ERROR;
\end{verbatim}
\end{description}

\newpage
\subsection{{\RTREE}-$>$cursor}
\begin{verbatim}
#include <rtree.h>
int RTree->cursor(RTree *rtree, RTreeCursor **cursorp, int flags);
\end{verbatim}
\begin{description}
\item[Description]\ \\

  The RTREE-$>$cursor function creates a cursor and copies a pointer
  to it into the memory referenced by ``cursorp''.

  A cursor is a structure used to provide sequential access through a
  database.
  
  Currently the flags value must be set to 0.

\item[Return Value]\ \\
  The RTREE-$>$cursor function returns a non-zero error value on
  failure and 0 on success.
\item[Sample Call]\ 
\begin{verbatim}
RTree *rtree;
RTreeCursor *rtc;

......

if((ret = rtree->cursor(rtree, &rtc, 0)) != 0)
    return ERROR;
\end{verbatim}
\end{description}

\newpage
\subsection{{\RTreeCursor}-$>$c\_get}
\begin{verbatim}
#include <rtree.h>
int RTreeCursor->c_get(RTreeC *cursor, char *tuple, size_t tplsz, int flags);
\end{verbatim}
\begin{description}
\item[Description]\ \\

  The flags parameter must be set to one of the following values: 
  \begin{description}
  \item[{\DBFIRST}]\ \\
    The cursor is set to reference the first tuple of the relation,
    and that pair is returned.

    If the database is empty, {\RTreeCursor}-$>$c\_get will return
    {\DBNOTFOUND}.
  \item[{\DBNEXT}]\ \\
    If the cursor is not yet initialized, {\DBNEXT} is identical to
    {\DBFIRST}.

    Otherwise, the cursor is moved to the next tuple of the database,
    and that pair is returned.

    If the cursor is already on the last record in the database,
    {\RTreeCursor}-$>$c\_get will return {\DBNOTFOUND}.
  \end{description}

\item[Return Value]\ \\
   If no matching keys are found, {\RTreeCursor}-$>$c\_get will return
   {\DBNOTFOUND}.  Otherwise, the {\RTreeCursor}-$>$c\_get function
   returns a non-zero error value on failure and 0 on success.

   If {\RTreeCursor}-$>$c\_get fails for any reason, the state of the
   cursor will be unchanged.

\item[Error]\ \\ 
  The {\RTreeCursor}-$>$c\_get function may fail and return a non-zero
  value -1 for disk access errors and memory exhaustion errors.

\item[Sample Call]\ 
\begin{verbatim}
RTreeCursor *rtc;
size_t tplsz = 6;
char *tuple=malloc(tplsz);

......

if((ret = rtc->c_get(rtc, tuple, 6, DB_NEXT)) == DB_NOTFOUND)
    .......
free(tuple);
\end{verbatim}
\end{description}

\newpage
\subsection{{\RTreeCursor}-$>$c\_del}
\begin{verbatim}
#include <rtree.h>
int RTreeCursor->c_del(RTreeC *cursorp, int flags);
\end{verbatim}
\begin{description}
\item[Description]\ \\

  The RTreeCursor-$>$c\_del function deletes the key/data pair currently
  referenced by the cursor. 

  The flags parameter is currently unused, and must be set to 0. 

\item[Return Value]\ \\
  If the cursor is not yet initialized or the database is
  currently empty, the RTreeCursor-$>$c\_del
  function will return {\DBNOTFOUND}. Otherwise, the
  RTreeCursor-$>$c\_del function returns 0 on success. 

\item[Sample Call]\ 
\begin{verbatim}
RTreeCursor *rtc;

......

if((ret = rtc->c_del(rtc, 0)) == DB_NOTFOUND)
   .......
\end{verbatim}
\end{description}

\newpage
\subsection{{\RTreeCursor}-$>$c\_close}
\begin{verbatim}
#include <rtree.h>
int RTreeCursor->c_close(RTreeCursor *cursorp);
\end{verbatim}
\begin{description}
\item[Description]\ \\

  The RTreeCursor-$>$c\_close function discards the cursor. 

  Once RTreeCursor-$>$c\_close has been called, regardless of its
  return, the cursor handle may not be used again. 

\item[Return Value]\ \\
  The RTreeCursor->c\_close function returns a non-zero error value
  on failure and 0 on success.

\item[Sample Call]\ 
\begin{verbatim}
RTreeCursor *rtc;

......

if((ret = rtc->c_close(rtc)) != 0)
    return ERROR;
\end{verbatim}
\end{description}

\newpage
\tableofcontents
\end{document}
