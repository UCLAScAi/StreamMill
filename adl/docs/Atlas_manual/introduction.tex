

\chapter{Introduction}

ATLaS is an SQL-based programming language for data-intensive applications.
Unlike languages, such as PL/SQL  or SQL/PSM, which use the imperative the constructs of
procedural languages, ATLaS achieves Turing completeness by using declarations, i.e.,
by supporting the declarations of new  aggregates and table functions.

An ATLaS program consists of list of declarations followed by a list of
SQL statements. Therefore, (with
{\tt A|B} denoting either {\tt A} or {\tt B}, and {\tt A*}
denoting zero or more occurrence of {\tt A}) we have:
%{\renewcommand{\baselinestretch}{1}
\begin{figure}[!htp]
\centering
\framebox{
\begin{tabular}{rll}
   \bnf{ATLaS-program} &$\rightarrow$& \bnf{ATLaS-dcl}$*$\;  \bnf{SQL-statement}$*$\\
  \bnf{ATLaS-dcl}     &$\rightarrow$&  \bnf{Table-dcl} $|$ \bnf{UDA-dcl} $|$ \bnf{TFunc-dcl} \\
  \end{tabular}
}
\caption{Syntax of ATLaS extensions \label{tab:syntax}}
\end{figure}

%}

The SQL-statements  mentioned  in Figure \ref{tab:syntax}
are the  basic select, insert, delete, update commands of
SQL-2,  which are summarized in Figure \ref{table:sql}.
We will assume that our reader is  already familiar with SQL-2 and
concentrate on the extensions which make
makes ATLaS so powerful.  ATLaS is Turing-complete
because of its declarations,
which are of three kinds:

\begin{enumerate}
\item  Table declarations (Table-dcl]),

\item  declarations of User-Defined Aggregates (UDA-dcl),

\item  declarations of Table Functions (TFunc-dcl).
\end{enumerate}

In the next  Section  we discuss simple programs that only use table
declarations. Programs
containing user-defined aggregates and table functions are
discussed in  the following sections.
