

\chapter{External Functions}
ATLaS supports both scalar external functions and table external
functions.

(For latest update, please refer to our website.)

\section{Scalar Functions}

To declare an external function to be dynamically loaded into ATLaS, we
use the following syntax:

\begin{verbatim}
external int ginif(a int) in 'gini.so';
\end{verbatim}

The above statement declares a UDF {\it ginif} which takes one
integer-type parameter and returns an integer result. This function is
supported by a shared library, 'gini.so'.

We can use C or any other language to create functions in shared
libraries. On a UNIX system, the following command compiles C source
code to dynamical libraries.

\begin{verbatim}
gcc -shared -o gini.so -fPIC gini.c
\end{verbatim}

Once defined, the UDF can be used in ATLaS. For instance:

\begin{verbatim}
select gini(a) from test;
\end{verbatim}

In order to dynamically load the library, the OS must be able to
find it. In UNIX, the OS searchs for the library in all the paths
specified by the environment variable LD\_LIBRARY\_PATH.

\section{Table Functions}
In much the same way, we can use external UDF as table functions.

For instance, we want to stream through the first K Fibonacci numbers.
It is not difficult to write a C function to generate the Fibonacci
numbers. The following ATLaS program demonstrates how to use such an
external table function.

\begin{verbatim}
external table (i int, f int) fib(k int) in 'tabf.so';

select t.i, t.f
from table (fib(10)) as t;
\end{verbatim}

In order to declare an external table function, we must use TABLE as
the return type. The above declaration indicates 'fib' is an external
function found in shared library 'tabf.so', and 'fib' returns a stream
of tubples (i,f), where f is the i-th Fibonacci number. Then, in the
following query, we stream through the first 10 Fibonacci numbers by
calling 'table (fib(10))'.

How do we implement a table function in C? Unlike stateless scalar
functions, table functions must keep their internal state between
calls. More specifically, the function must be able to: i) determine
the first call from subsequent calls; ii) tell the caller whether a
tuple is successfully returned; iii) use a mechanism to return tuples
to the caller. As an example, the following code implements function
'fib':

\begin{verbatim}
#include <db.h>

struct result {
  int a;
  int b;
};

int fib(int first_entry, struct result *tuple, int k)
{
  static int count;
  static int last;
  static int next;

  if (first_entry == 1) {
    count = 0;
    next=1;
    last=0;
  }
  if (count++ <k) {
    tuple->a = count;
    tuple->b = last;

    last = next;
    next = next+tuple->b;
    return 0;
  } else {
    return DB_NOTFOUND;
  }
}
\end{verbatim}

In addition to the arguments (here is 'k') passed to the table
function, we have 2 extra arguments: i) $first\_entry$, if
first\_entry=1 then it is the first call; ii) tuple, which is a
pointer to a structure where results are to be stored. External
table functions always return an integer value, 0 if successful,
DB\_NOTFOUND otherwise.

A possible use of table functions is to scan file system data, and
return results to the database system after filtering. Our test
indicates that on a linux system, external table functions accessing
file system data is almost 100 times faster than accessing the same
data in the Berkeley DB format.


\section{Built-in Aggregates and Functions}
ATLaS supports the standard builtin aggregates: {\bf min()}, {\bf
  max()}, {\bf sum()}, {\bf avg()}, and {\bf count()}.

ATLaS supports the following builtin functions: (they are being added
constantly.)

\begin{itemize}
\item {\bf srand(INT) : INT} \\
  The srand() function sets its argument as the seed for a new
  sequence of pseudo-random integers to be returned by rand().  These
  sequences are repeatable by calling srand() with the same seed
  value. srand() always returns 0.

\item {\bf rand() : REAL} \\
  The rand() function returns a pseudo-random real between 0 and 1.
  The following code set 10 as a random seed, and displays two random
  values.
\begin{verbatim}
    VALUES(srand(10));

    VALUES(rand(), rand());
\end{verbatim}

\item {\bf sqrt(REAL) : REAL}\\
  The sqrt(x) function returns the non-negative square root of x.  \\

\item {\bf timeofday() : CHAR(20)}\\
  The gettimeofday function gets the system's notion of the current
  time. The current time is expressed in elapsed seconds and
  microseconds since 00:00 Universal Coordinated Time, January 1,
  1970. It returns a string in the form of x'y'', where x is the
  seconds and y is the microseconds. This function is maily used to
  measure the performance of ATLaS queries, as in the following example:

\begin{verbatim}
    INSERT INTO stdout VALUES(timeofday());

    ... some ATLaS queries ...

    INSERT INTO stdout VALUES(timeofday());
\end{verbatim}
\end{itemize}
