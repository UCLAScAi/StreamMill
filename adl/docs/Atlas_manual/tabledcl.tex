\chapter{ATLaS SQL on Tables}
ATLaS can use a variety of tables from different sources. In
particular, it can access B+Tree indexed tables managed by the embedded
database system Berkeley DB~\cite{berkeleydb}. For instance, say that an
employee table, with key $\tt Eno$, is stored in such a format in
the directory\\
$\tt C$:$\tt \backslash mydb \backslash employees$.
[Say more about win vs linux] Then the following
program  first gives a 5 \% raise to the employees in the 'QA'
department, and then prints all the employees who now make more
than 60K:


\begin{codedisplay}
\tt
\>\>\>\>\>\>\>\>\> $\slash$* Begin of ATlaS program--- this is a comment*$\slash$\\
\>\>table employees(Eno int, Name char(18), Sal real, Dept char(6)) \\
 \> \>    \> \> \> \> Btree(Eno) ~  source \
 '$\tt C$:$\tt \backslash  mydb\backslash employees$';\\[0.1cm]
 \> \> update employees set Sal = Sal * 1.05\\
       \> \> \> \> \> where Dept='QA' ;\\[0.1cm]
 \> \> insert into stdout select Eno, Name\\
 \> \> \> \> \> from employees where Sal $>$ 60000;\\
\>\>\>\>\>\>\>\>\> $\slash$* End of ATlaS program *$\slash$\\
\end{codedisplay}

The {\tt insert into stdout} clause before {\tt select} in the last
statement can be omitted without changing the meaning of this program.
Thus ATLaS supports the standard {\tt select, insert, delete, update}
statements of SQL-2. Observe that keywords can be written in upper
case or lower case---however, attribute names and other user-defined
identifiers are case-sensitive.

Since we assume that our readers are already familiar
with SQL-2, we will now concentrate on the new constructs of ATLaS and
return to SQL in Section 6.

Three types of tables are currently supported in ATLaS, as follows:

\begin{enumerate}
\item Secondary storage tables with indexed by B+ trees on one or more attributes,
as in the following example:

\begin{codedisplay}
\tt
\>\> TABLE employees(Eno int, Name char(18), Sal real, Dept char(6)) \\
 \> \>  \> \> \> \> BTree(Eno)
 source '$\tt C$:$\tt \backslash  mydb \backslash employees$';\\
\end{codedisplay}

The  $\tt SOURCE$ declaration associates this table with the file
$\tt C$:$\tt \backslash  mydb \backslash employees$ and make it
 {\em persistent}. Persistent tables remain after
the the program completes its execution. Tables declared without a $\tt source$
are {\em transitory} and they are removed at the end of the program
execution.

\item Secondary storage tables with indexed by R+ trees on a pair
or a quadruplet of real attributes: the first can be
used to index points, and the second for rectangles:

\begin{codedisplay}
\tt
\>TABLE mypoints(x real, y real, object char(10)) RTree(x,y); \\[0.2cm]

\> TABLE myrectangles(tx real, ty real, bx real, by real, object char(10))\\
  \> \>  \> \> \> \>      \> \>  \> \> \> \>     RTree(tx,ty,bx,by);
\end{codedisplay}
\item
Main memory tables, with a hash-based index on one or more
attributes:
\begin{codedisplay}
\tt
TABLE memo(j Int,  Region Char(20))\\
\>\>\> MEMORY AS VALUES(0, `root-of-tree`);
\end{codedisplay}
This example also illustrates that a transitory
table can be  initialized to the results of the query defined using the
{\tt AS query} option. In this example, we use constant values to
initialize the table. In general, the initialization query can
use the content of previously declared tables.
\end{enumerate}
A source declaration can only be given for B+tree and R+tree tables, but not for
{\tt MEMORY} tables since these are never persistent.
The syntax of table declarations is as follows:


\begin{table}[!htp]
\centering
\framebox{
\begin{tabular}{rll}
    \bnf{Table-dcl} &$\rightarrow$ & `TABLE` ~ \bnf{table-id}
`(` \bnf{columns} \bnf{keydec} `)`\\
     & &~~~~~~~~~~ [`SOURCE` $|$ `$'$`\bnf{file-name}`$'$``AS` \bnf{query}] `;`\\
    \bnf{column-list} &$\rightarrow$& \bnf{column} [`,`  \bnf{column}]* \\
    \bnf{column} &$\rightarrow$& \bnf{id} \bnf{type} \\
    \bnf{type} &$\rightarrow$& `INT` $|$ `REAL` $|$ `CHAR` `(` \bnf{num} `)` $|$  `REF` `(` \bnf{id} `)`\\
    \bnf{keydec} &$\rightarrow$& [`BTree(`\bnf{key}`)`, `RTree(`\bnf{key}`)`, MEMORY] `;`\\
  \bnf{key} &$\rightarrow$&  `(` \bnf{id} [`,` \bnf{id}]* `)` \\
 \\
 \bnf{load} &$\rightarrow$& `LOAD` \ `FROM` `$'$`\bnf{file-name}`$'$` `INTO` \bnf{table-id}\\

  \end{tabular}
}
\caption{Declaring and Initializing Tables in ATLaS.}
\end{table}
The {\tt LOAD} construct of ATLaS  can be used to load into a table data from
an external file, where the commas (carriage returns) are used as separators
between attribute values (records). Newly loaded tuples are appended to
the existing tuples.
