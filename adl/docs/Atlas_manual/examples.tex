\chapter{Programming in ATLaS}
\section{Recursion}
Example \ref{exe:trc} illustrates the typical structure of an ATLaS
program. The declaration of the table {\bw dgraph(start, end)} is followed
by that of the UDA {{\bw reachable}}; the table and the UDA are then used
in an SQL statement that calls for the computation of all nodes reachable from
node {\bw '000'}. The clause {\cw SOURCE({\bw mydb})} denotes that
{\bw dgraph} is a table from a database that is known to
the systems by the name {\bw 'mydb'}. (Without the {\kw{SOURCE}} clause
{\bw dgraph} is local to the program and discarded once its execution
 completes).

The program of Example~\ref{exe:trc}
shows one way in which the transitive closure
of  a graph can be expressed in
ATLaS. We use the recursive UDA {{\bw reachable}}
that perform a depth-first traversal of the graph
by recursively calling itself.  Upon receiving a node,
{{\bw reachable}} returns the node
to the calling query along with all the nodes reachable from it.
(We assume that the graph contains no directed cycle;
otherwise a table will be needed to memorize
previous results and avoid
infinite loops.)
Observe that the {\cw INITIALIZE} and {\cw ITERATE} routine in
Example~\ref{exe:trc} share the same block of code.


\begin{example}{Computation of Transitive Closure\label{exe:trc} \pinv}

\begin{codedisplay}
\>\>\kw{TABLE} dgraph(start \kw{Char}(10), end \kw{Char}(10))  \kw{SOURCE} (mydb); \\[0.2CM]

\>\>\kw{AGGREGATE} reachable(Rnode \kw{Char}(10)) : \kw{Char}(10) \\
\>\>\{\> \kw{INITIALIZE}: \ \kw{ITERATE}: \{ \\
\>\>\>\> \kw{INSERT} \kw{INTO} \kw{RETURN} \kw{VALUES} (Rnode) \\
\>\>\>\>\kw{INSERT} \kw{INTO} \kw{RETURN} \\
\>\>\>\>\> \kw{SELECT} reachable(end) \kw{FROM} dgraph\\
\>\>\>\>\> \kw{WHERE}  start=Rnode;  \\
\>\>\>\}\\
\>\>\}\\
\>\>\kw{SELECT} reachable(dgraph.end) \kw{FROM} dgraph \\
\>\>\kw{WHERE} dgraph.start='000';
\end{codedisplay}

\end{example}

Besides the  Prolog-like top-down
computation of Example \ref{exe:trc}, we can also express easily the
bottom-up computation used in  Datalog,
and a stream-oriented computation will be discussed in the next section.

Recursive queries can
be supported in  ATLaS without any
new construct since UDAs can call other
aggregates or call themselves recursively. Examples of
application of recursive UDAs in data mining will be discussed
later.

Along with recursive aggregates, table functions defined in SQL play a
critical role in expressing data mining applications in ATLaS.  For
instance, let us consider the table function {\bw dissemble} that will
be used to express decision tree classifiers.  Take for instance
the well-known Play-Tennis example of Figure \ref{tab:newtennis}; here we
want to classify the value of {\bw Play} as a `Yes' or a `No' given a
training set such as that shown in Table~\ref{tab:newtennis}.

{\renewcommand{\baselinestretch}{1}
\normalsize
\begin{figure}[htb]
\begin{center}
{\footnotesize
\begin{tabular}{|c|l|l|l|l|c|} \hline
{\bf RID}&{\bf Outlook}&{\bf Temp}&{\bf Humidity}&{\bf Wind}&{\bf Play}\\
\hline
1&Sunny&Hot&High&Weak&No\\
2&Sunny&Hot&High&Strong&No\\
3&Overcast&Hot&High&Weak&Yes\\
4&Rain&Mild&High&Weak&Yes\\
5&Rain&Cool&Normal&Weak&Yes\\
6&Rain&Cool&Normal&Strong&Yes\\
7&Overcast&Cool&Normal&Strong&No\\
8&Sunny&Mild&High&Weak&No\\
9&Sunny&Cool&Normal&Weak&Yes\\
10&Rain&Mild&Normal&Weak&Yes\\
11&Sunny&Mild&Normal&Strong&Yes\\
12&Overcast&Mild&High&Strong&Yes\\
13&Overcast&Hot&Normal&Weak&Yes\\
14&Rain&Mild&High&Strong&No\\ \hline
\end{tabular}
}
\caption{The relation \bf PlayTennis\inv \inv} \label{tab:newtennis}
\end{center}
\end{figure}
}

The first step for most scalable classifiers~\cite{sprint} is to
convert the training set into column/value pairs. This conversion,
although conceptually simple, is hard to express succinctly in SQL.
Consider the {\bw PlayTennis} relation as shown in Figure \ref{tab:newtennis}. We want to convert {\bw PlayTennis} into a new
stream of three columns {\bw (Col, Value, YorN)} by breaking down each
tuple into four records, each record representing one column in the
original tuple, including the column number, column value and the
class label {\bw YorN}.  We can define the table function
{\bw dissemble} of Example \ref{exe:revert} to solve the problem.
Then, using this table function and the recursive aggregate {\bw
classify}, Algorithm~\ref{alg:sprint} implements a
scalable decision tree classifier using
merely 15 statements.
%\begin{figure}[b]
\begin{example}{Dissemble a relation into column/value pairs.}

\begin{codedisplay}
\kw{FUNCTION} dissemble (v1 \kw{Int}, v2 \kw{Int}, v3 \kw{Int},
v4 \kw{Int}, YorN \kw{Int}):\\
\>\>\>\>\>\>\>\>(col \kw{Int}, val \kw{Int}, YorN \kw{Int}); \\[0.2cm]
%\> \>  \> \>\kw{}: (col \kw{Int}, val \kw{Int}, YorN \kw{Int}); \\
 \{\>\kw{INSERT} \kw{INTO} \kw{RETURN} \kw{VALUES} \\
 \>\>\>(1, v1, YorN), (2, v2, YorN), \kw{}(3, v3, YorN), (4, v4, YorN);
\}
\end{codedisplay}
 \vspace*{-0.2cm}
\label{exe:revert}
\end{example}

{\renewcommand{\baselinestretch}{1}
\normalsize

\begin{algorithm}[!htb]
\begin{algorithmic}[1]
\STATE\kw{AGGREGATE} classify(iNode \kw{Int}, RecId \kw{Int}, iCol \kw{Int}, \kw{}iValue \kw{Int}, iYorN \kw{Int})\\
\STATE\{\hspace{.2cm}\kw{TABLE} treenodes(RecId \kw{Int}, Node \kw{Int}, Col \kw{Int}, \kw{}Value \kw{Int}, YorN \kw{Int});\\
\STATE\hspace{.3cm}\kw{TABLE} mincol(Col \kw{Int});\\
%\STATE\hspace{.3cm}\kw{TABLE} summary(Col \kw{Int}, Value \kw{Int}, Yc \kw{Int}, Nc \kw{Int},\\
%\hspace{2.5cm}\kw{INDEX} \{Col,Value\});\\
\STATE\hspace{.3cm}\kw{TABLE} summary(Col \kw{Int}, Value \kw{Int}, Yc \kw{Int}, Nc \kw{Int})
\kw{BTree}(Col,Value);\\
\STATE\hspace{.3cm}\kw{TABLE} ginitable(Col \kw{Int}, Gini \kw{Int});\\
\STATE\hspace{.3cm}\kw{INITIALIZE} : \kw{ITERATE} : \{\\
\STATE\hspace{.6cm}\kw{INSERT} \kw{INTO} treenodes \kw{VALUES}(RecId, iNode, iCol, iValue, iYorN);\\
\STATE\hspace{.6cm}\kw{UPDATE} summary\\
\hspace{.9cm}\kw{SET} Yc=Yc+iYorN, Nc=Nc+1-iYorN\\
\hspace{.9cm}\kw{WHERE} Col = iCol \kw{AND} Value = iValue;\\
\STATE\hspace{.6cm}\kw{INSERT} \kw{INTO} summary \\
\hspace{.9cm}\kw{SELECT} iCol, iValue, iYorN, 1-iYorN \\
\hspace{.9cm}\kw{WHERE} SQLCODE > 0;\\
\hspace{.3cm}\}\\
\STATE\hspace{.3cm}\kw{TERMINATE} : \{\\
\STATE\hspace{.6cm}\kw{INSERT} \kw{INTO} ginitable\\
\hspace{.9cm}\kw{SELECT} Col, sum((Yc*Nc)/(Yc+Nc))/sum(Yc+Nc) \kw{FROM} summary\\
\hspace{.9cm}\kw{GROUP BY} Col \kw{HAVING} count(Value)$>$1 \kw{AND} sum(Yc)$>$0 \kw{AND} sum(Nc)$>$0;\\
\STATE\alglabel{sprintmincolumn}\hspace{.6cm}\kw{INSERT} \kw{INTO} mincol \kw{SELECT} minpair(Col, Gini)$\rightarrow$mPoint \kw{FROM} ginitable;\\
\STATE\hspace{.6cm}\kw{INSERT} \kw{INTO} result \kw{SELECT} iNode, Col \kw{FROM} mincol;\\
\hspace{.6cm}\COMMENT{Call classify() recusively to partition each of its subnodes unless it is pure.}\\
\STATE\hspace{.6cm}\kw{SELECT} classify(t.Node*MAXVALUE+m.Value+1, t.RecId, t.Col, t.Value, t.YorN)\\
\hspace{.9cm}\kw{FROM}\ \ \ treenodes \kw{AS} t,\\
\hspace{1.2cm}( \kw{SELECT} tt.RecId RecId, tt.Value Value\\
\hspace{1.3cm}  \kw{FROM} treenodes \kw{AS} tt, mincol \kw{AS} m\\
\hspace{1.3cm}  \kw{WHERE} tt.Col=m.Col; \\
\hspace{1.2cm}\kw{}) \kw{AS} m\\
\hspace{.9cm}\kw{WHERE} t.RecId = m.RecId AND\\
\hspace{2.1cm}t.col \kw{NOT IN} (\kw{SELECT} col \kw{FROM} mincol)\\
% \hspace{1.3cm}\kw{AND EXIST} (\kw{SELECT} Col \kw{FROM} summary\\
% \hspace{2.9cm}\kw{GROUP BY} Col\\
% \hspace{2.9cm}\kw{HAVING} count(Value)$>$1 \\
% \hspace{3.4cm}\kw{AND} sum(Yc) $>$ 0\\
% \hspace{3.4cm}\kw{AND} sum(Nc) $>$ 0) \\
\alglabel{sprintgroup}\hspace{.9cm}\kw{GROUP BY} m.Value;\\
\hspace{.3cm}\}\\
\STATE\}
\end{algorithmic}
\caption{A Scalable Decision Tree Classifier}
\label{alg:sprint}
\end{algorithm}


In the {\cw INITIALIZE} and {\cw ITERATE} statements of {\bw classify}
in Algorithm \ref{alg:sprint}, we update the class histogram kept in
the {\bw summary} table for each column/value pair.  The {\cw
  TERMINATE} routine first computes the gini index for each column
using the histogram.  However, if a column has a single value ({\small
  \bf count(Value)$\le$ 1}), or tuples in the partition belongs to one
class ({\small \bf{sum(Yc)=0}} or {\small \bf{sum(Nc)=0}}), then the
column is not splittable and hence, not inserted into {\bw ginitable}.
On line~\algref{sprintmincolumn}, we select the splitting column which
has the minimal gini index. A new sub-branch is generated for each
value in the column.  The UDA {\bw minpair} previously defined
 returns the minimal gini index as well
as the column where the minimum value occurred. After recording the
current split into the {\bw result} table, we call the classifier
recursively to further classify the sub nodes.  On
line~\algref{sprintgroup}, {\cw GROUP BY} {\bw m.Value} partitions the
records in {\bw treenodes} into {\bw MAXVALUE} subnodes, where {\bw
  MAXVALUE} is the largest number of different values in any of the
table columns (three for Figure \ref{tab:newtennis}).
% The {\cw EXIST} subquery in the {\cw WHERE} clause checks the
% termination condition.
The recursion terminates if table {\bw mincol} is empty, that is,
there is no valid column to further split the partition.

To classify the PlayTennis dataset shown in Table~\ref{tab:newtennis}, we
use the program:
\begin{codedisplay}
\>\>\kw{SELECT} classify(0, p.RID, d.Col, d.Val, d.YorN)\\
\>\>\kw{FROM} PlayTennis \kw{AS} p, \\
\>\>\>\>\kw{TABLE}(dissemble(Outlook,Temp, Humidity, Wind, Play)) \kw{AS} d;\\
\end{codedisplay}
%where {\bw dissemble} is the table function of Example~\ref{exe:revert})
%that converts the PlayTennis table into column/value pairs.
% and p.RID is the (pseudo) ROWID column of the PlayTennis table.
%%%%%%%%%%%%%%%%%%%%%%%%%%%%%%%%%%%%%%%
%\section{Recursion}

Table functions and recursion are also supported in SQL 1999, but, at
the best of our knowledge, there is no simple way to express
decision-tree classifiers in SQL (or for that matter in Datalog).
The fact that a concise expression for this algorithm is now
possible suggests the stream-oriented
computation model of UDAs adds to the expressive power of ATLaS.

%%%%%%%%%%%%%%%%%%%%%%%%%%%%%%%%%%%%%%%%%%%%%%%

\section{ROLAPs}

Powerful aggregate extensions based on modifications and
generalization of group-by constructs have recently been proposed by
researchers, OLAP vendors, and standard committees. New operators,
such as % {\bw GROUPING SETS}
{\cw ROLLUP} and {\cw CUBE}, have been included in SQL-3 and
implemented in major
commercial DBMSs.
We will now express these extensions
in ATLaS.

The purpose of {\cw ROLLUP} is to create subtotals at multiple
detail levels from the most detailed one, up to the grand total. This
functionality could  be expressed in basic SQL
by combining several {\cw SELECT}
statements with {\cw UNION}s. For instance, to roll up a sales table
along dimensions such as Time, Region, and Department, we can use the
query of Example \ref{exe:sqlrollup}.

% {\renewcommand{\baselinestretch}{1}
% \normalsize
% \footnotesize
% \begin{codedisplay}
% \>\>\kw{SELECT} Time, Region, Dept, sum(Sales) \kw{AS} Sales \\
% \>\>\kw{FROM} data\\
% \>\>\kw{GROUP BY ROLLUP}(Time, Region, Dept)\\
% \end{codedisplay}
% }
%\begin{figure} [t]
{\small
\begin{example}{Using Basic SQL to express ROLLUP}
\begin{codedisplay}
\kw{SELECT} Time, Region, Dept, SUM(Sales) \kw{FROM} Sales \kw{GROUP BY} Time, Region, Dept\\
\>\>\>\>\kw{UNION ALL}\\
\kw{SELECT} Time, Region, `all' , SUM(Sales) \kw{FROM} Sales \kw{GROUP BY} Time, Region\\
\>\>\>\>\kw{UNION ALL}\\
 \kw{SELECT} Time, `all', `all', SUM(Sales) \kw{FROM} Sales \kw{GROUP BY} Time\\
\>\>\>\>\kw{UNION ALL}\\
\kw{SELECT} `all', `all', `all', SUM(Sales) \kw{FROM} Sales;
\end{codedisplay}
\label{exe:sqlrollup}
\end{example}
}
%\end{figure}

The problem with the approach in Example~\ref{exe:sqlrollup}, above, is
that each of the four {\cw SELECT} statements could result in a new scan
of the table, even though all needed subtotals can be gathered in a
single pass. Thus, a new {\cw ROLLUP} construct was introduced in SQL.

No ad hoc operator is needed in ATLaS to express rollup queries.
For instance, in ATLaS the above query can be expressed succinctly as
follows:
%\footnote{We assume the dataset is ordered by Time, Region, and Dept.}:
%\newpage
{
\begin{codedisplay}
\> \> \kw{SELECT} rollup(Time, Region, Dept, Sales) \kw{FROM} data;\\
\end{codedisplay}
}
Indeed, the rollup functionality can be expressed by ATLaS in several
different ways. Algorithm~\ref{alg:rollup} shows an implementation of
the {\bw rollup} aggregate used in the above query, where the dataset
is assumed ordered by Time, Region, and Dept.
{
  \renewcommand{\baselinestretch}{1}
  \normalsize

  \begin{algorithm}[!htb]
    \begin{algorithmic}[1]
      \footnotesize
      \STATE\kw{AGGEGATE} rollup(T \kw{Char}(20), R \kw{Char}(20), \kw{}D \kw{Char}(20), Sales \kw{Real}): (T \kw{Char}(20), R \kw{Char}(20), D \kw{Char}(20), Sales \kw{Real})\\
      \STATE\{\hspace{.15cm}\kw{TABLE} memo(j \kw{Int}, Time \kw{Char}(20), Region \kw{Char}(20), \kw{}Dept \kw{Char}(20), Sum \kw{Real}) \kw{MEMORY};\\
      \STATE\hspace{.3cm}\kw{FUNCTION} onestep(L \kw{Int}, T \kw{Char}(20), R \kw{Char}(20), \kw{}D \kw{Char}(20), S \kw{Real})\\
      \hspace{.8cm}: (T \kw{Char}(20), R \kw{Char}(20), D \kw{Char}(20), Sales \kw{Real})\\
      \STATE\hspace{.3cm}\{\hspace{.2cm}\kw{INSERT} \kw{INTO} \kw{RETURN} \\
                                %\STATE\hspace{1cm}\kw{SELECT} TY, TM, TD \kw{FROM} memo\\
      \hspace{.8cm}\kw{SELECT} Time, Region, Dept, Sum \kw{FROM} memo\\
      \hspace{.8cm}\kw{WHERE} L=j \kw{AND} (T$\neq$Time \kw{OR} R$\neq$Region \kw{OR} D$\neq$Dept);\\
                                %\hspace{1cm}\\
      \STATE\hspace{.6cm}\kw{UPDATE} memo \kw{SET} Sum = Sum + (\kw{SELECT} m.Sum \kw{FROM} memo \kw{AS} m \kw{WHERE} m.j=L)\\
      \hspace{.8cm}\kw{WHERE} SQLCODE=0 \kw{AND} j=L+1;\\
                                %\hspace{1cm}\\
      \STATE\hspace{.6cm}\kw{UPDATE} memo\\
      \hspace{.8cm}\kw{SET} Time=T, Region=R, Dept=D, Sum=S\\
      \hspace{.8cm}\kw{WHERE} SQLCODE=10 \kw{AND} j=L;\\
      \hspace{.3cm}\} \\
      \STATE\hspace{.3cm}\kw{INITIALIZE}: \{ \\
      \STATE\alglabel{rollupinit}\hspace{.6cm}\kw{INSERT} \kw{INTO} memo \\
      \hspace{.8cm}\kw{VALUES} (1, T, R, D, Sales), (2, T, R, `all', 0), (3, T, `all', `all', 0), (4, `all', `all', `all', 0);\\
      \hspace{.3cm}\}\\
      \STATE\hspace{.3cm}\kw{ITERATE}: \{\ % if there is no change in the input\\
      \STATE\alglabel{rollupadd}\hspace{.6cm}\kw{UPDATE} memo \kw{SET} Sum = Sum + Sales \kw{WHERE} Time=T \kw{AND} Region=R \kw{AND} Dept=D;\\
                                %\STATE\hspace{1cm}\ \ \ \  %otherwise\\
      \STATE\alglabel{rollupstep}\hspace{.6cm}\kw{INSERT} \kw{INTO} \kw{RETURN} \kw{SELECT} os.* \kw{FROM} \kw{TABLE}(onestep(1, T, R, D, Sales)) \kw{AS} os\\
      \hspace{.8cm}\kw{WHERE} SQLCODE>0;\\
                                %\STATE\hspace{1cm}\ \ \ \ \ \  %and if the previous is true\\
      \STATE\hspace{.6cm}\kw{INSERT} \kw{INTO} \kw{RETURN} \kw{SELECT} os.* \kw{FROM} \kw{TABLE}(onestep(2, T, R, `all', 0)) \kw{AS} os\\
      \hspace{.8cm}\kw{WHERE} SQLCODE>0;\\
                                %\STATE\hspace{1cm}\ \ \ \ \ \  %and if the previous is true\\
      \STATE\hspace{.6cm}\kw{INSERT} \kw{INTO} \kw{RETURN} \kw{SELECT} os.* \kw{FROM} \kw{TABLE}(onestep(3, T, `all', `all', 0)) \kw{AS} os\\
      \hspace{.8cm}\kw{WHERE} SQLCODE=1;\\
      \hspace{.3cm}\}\\
      \STATE\hspace{.3cm}\kw{TERMINATE}: \{\\
      \STATE\hspace{.6cm}\kw{INSERT} \kw{INTO} \kw{RETURN} \kw{SELECT} os.* \kw{FROM} \kw{TABLE}(onestep(1, `all', `all', `all', 0)) \kw{AS} os;\\
      \STATE\hspace{.6cm}\kw{INSERT} \kw{INTO} \kw{RETURN} \kw{SELECT} os.* \kw{FROM} \kw{TABLE}(onestep(2, `all', `all', `all', 0)) \kw{AS} os;\\
      \STATE\hspace{.6cm}\kw{INSERT} \kw{INTO} \kw{RETURN} \kw{SELECT} os.* \kw{FROM} \kw{TABLE}(onestep(3, `all', `all', `all', 0)) \kw{AS} os;\\
      \STATE\hspace{.6cm}\kw{INSERT} \kw{INTO} \kw{RETURN} \kw{SELECT} Time, Region, Dept, Sum \kw{FROM} memo \kw{WHERE} j=4;\\
      \hspace{.3cm}\}\\
      \STATE\}\\
    \end{algorithmic}
    \caption{Roll-up  sales on Time, Region, Dept\label{alg:rollup}}
  \end{algorithm}
  }
% {\renewcommand{\baselinestretch}{1}
% \normalsize
% \begin{algorithm}[htb]
% \begin{algorithmic}[1]
% \footnotesize
% \STATE\kw{FUNCTION} onestep(L \kw{Int}, T \kw{Char}(20), R \kw{Char}(20), D \kw{Char}(20), S \kw{Real}) : \\
% \hspace{2.5cm}(T \kw{Char}(20), R \kw{Char}(20), D \kw{Char}(20), Sales \kw{Real})\\
% \STATE\{\\
% \STATE\hspace{.5cm}\kw{INSERT} \kw{INTO} \kw{RETURN} \\
% %\STATE\hspace{1cm}\kw{SELECT} TY, TM, TD \kw{FROM} memo\\
% \hspace{.5cm}\kw{SELECT} Time, Region, Dept \kw{FROM} memo\\
% \hspace{.5cm}\kw{WHERE} L=level \kw{AND} (T$<>$Time \kw{OR} R$<>$Region \kw{OR} D$<>$Dept);\\
% %\hspace{1cm}\\
% \STATE\hspace{.5cm}\kw{UPDATE} memo \kw{SET} Sum= Sum + (\kw{SELECT} m.Sum \kw{FROM} memo \kw{AS} m \kw{WHERE} m.level=L)\\
% \hspace{.5cm}\kw{WHERE} SQLCODE=1 \kw{AND} level=L+1;\\
% %\hspace{1cm}\\
% \STATE\hspace{.5cm}\kw{UPDATE} memo \kw{SET} Time=T, Region=R, Dept=D, Sum=S\\
% \hspace{.5cm}\kw{WHERE} SQLCODE=1 \kw{AND} level=L;\\
% \STATE\} \\
% \end{algorithmic}
% \caption{Update table {\bw memo} starting from a certain level}
% \label{alg:onestep}
% \end{algorithm}
% }
Algorithm~\ref{alg:rollup} combines UDAs and table functions to
implement {\bw rollup}.

We use a  in-memory table to keep track of the
subtotals at each rollup level $j$ ($j$ = 1, ..., 4, with level 4
corresponding to the grand total). The table is as follows:

{\small \begin{codedisplay}
\kw{}TABLE memo(j \kw{Int}, Time \kw{Char}(20), Region \kw{Char}(20), Dept \kw{Char}(20), Sum \kw{Real})\\
  \> MEMORY
\end{codedisplay}}

At the core of the algorithm, we have the four entries added to {\bw
  memo} by the {\cw INITIALIZE} statement (line~\algref{rollupinit}).
The first entry has rollup level one and its subtotal (last column) is
initialized to the sales amount of the first record.  The subtotals
for the other three entries are set to zero.  Let  $memo_j$ denote
the memo tuple at level $j$; then, $memo_j$ contains $j-1$ occurrences of {\bw
  `all'}.

The four SQL statements in the {\cw ITERATE} group (i) determine the
rollup levels to which the next tuple in the input will contribute,
and (ii) for each such level, return values, and update the memo
table.  For instance, if the three leftmost columns of the new input
tuple match those of $memo_1$, then the new input value is also of
level one. If this is not the case, but the two leftmost columns match
those of $memo_2$, then the new tuple is of level two, and so on.  If
one (none) of the columns matches, then the new tuple is considered to be of
level 3 (level 4).  For incoming tuples at level 1, we update the subtotal for
$memo_1$ but return no result. For tuples of level 2 (level 3), we return the
current subtotal from $memo_1$ (and $memo_2$), reset this subtotal using the input
value, and then update the subtotal at $memo_2$($memo_3$).  For tuples belonging
to level $4$, we return the subtotals from $memo_j, j=1,2,3$ and reset
them to new input value.  The {\cw TERMINATE} statement returns the
subtotals from $memo_j, j=1,2,3,4$ where $memo_4$ now contains the sum
of all sales.

This computation is implemented by Algorithm~\ref{alg:rollup} with the
help of a special variable of SQL, {\cw SQLCODE}.  If no updates are made on
line~\algref{rollupadd}, i.e., if {\cw SQLCODE>0}, we need to use {\bw
  onestep()} to ``roll up'' the subtotals from level 1 to level 2
(line~\algref{rollupstep}). If the roll-up is successful, then we need
to check if further roll-ups from level 2 to level 3, and then from
level 3 to level 4 are necessary.

% In the {\bw TERMINATE} routine, we force a final roll-up on each level
% of aggregation and returns all the remaining subtotals as well as the
% grand total.

The table function {\bw onestep} is rather simple. We first test if
the level of the record being passed is different from the entry
$memo_j$ (to simplify this test some of its columns are conveniently
set to {\bw all}).  If they are different, then the subtotal for
stored in $memo_j$ must be returned.  This subtotal must also be
passed ('rolled-up') to the next level: i.e., to level $j+1$.
Finally, the subtotal at $memo_j$ must be reset from current input
record to restart aggregation on a new set of group-by columns.


In Algorithm~\ref{alg:rollup}, we assumed that the data is already
sorted on the rollup columns. When this is not the case, then
we use the  UDA $\bw sort\_and\_roll$, of Algorithm \ref{alg:sortandroll},
which first sort the data and then calls the UDA  {\cdf rollup}.
Furthermore, the {\cw CUBE} operator is
easily implemented  by a sequence of three sort-and-roll.

In Algorithm~\ref{alg:rollup},  {\cdf sort\_temp}
applies the standard SQL clause {\cw ORDER BY} to the output
tuples. Clearly, an SQL statement that contains an
 {\cw ORDER BY} clause is blocking, since it requires sorting.
Therefore, the table function {\bw sort\_temp} is also blocking since it
contains this statement. Thus, table functions can become blocking
because they contain SQL-statements with {\cw ORDER BY}, or
blocking aggregates or blocking table functions; but, except for those
situations, table functions are nonblocking.

{\renewcommand{\baselinestretch}{1}
\normalsize
\begin{algorithm}[tb]
%\begin{algorithm}[!htb]
\begin{algorithmic}[1]
\STATE\kw{AGGREGATE} sort\_and\_roll(T \kw{Char}(20), R \kw{Char}(20), \kw{}D \kw{Char}(20), Sales \kw{Real})\\
\hspace{.6cm}: (T \kw{Char}(20), R \kw{Char}(20), D \kw{Char}(20), Sales \kw{Real})\\
%
\STATE\{\hspace{.05cm}\kw{TABLE} temp(A1 \kw{Char}(20),
A2 \kw{Char}(20), A3 \kw{Char}(20),\kw{}V \kw{Real}) \kw{MEMORY};\\
\STATE\hspace{.2cm}\kw{FUNCTION} sort\_temp(): (A1 \kw{Char}(20), A2 \kw{Char}(20), A3 \kw{Char}(20), V \kw{Real})\\
\STATE\hspace{.2cm}\{\hspace{.2cm}\kw{INSERT INTO RETURN}\\
\hspace{.9cm}\kw{SELECT} * \kw{FROM} temp \kw{ORDER BY} A1, A2, A3;\\
\hspace{.2cm}\}
\STATE\hspace{.2cm}\kw{INITIALIZE}: \kw{ITERATE}: \{\\
\hspace{.6cm}\kw{INSERT} \kw{INTO} temp \kw{VALUES} (T, R, D, Sales);\\
\hspace{.2cm}\} \\
\STATE\hspace{.2cm}\kw{TERMINATE}: \{\\
\STATE\hspace{.6cm}\kw{INSERT} \kw{INTO} \kw{RETURN} \kw{SELECT} rollup(t.A1, t.A2, t.A3, t.V) \kw{FROM} \kw{TABLE} (sort\_temp()) \kw{AS} t;\\
\hspace{.2cm}\}\\
\STATE\}\\
\end{algorithmic}
\caption{Sorting and then rolling-up sales on Time, Region, Dept}
\label{alg:sortandroll}
\end{algorithm}
}
