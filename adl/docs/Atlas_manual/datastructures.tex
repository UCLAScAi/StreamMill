
\section{References and Data Structures}

In ATLaS, the types of table columns and the types of parameters to
User-Defined aggregates can be any of the following:

\begin{itemize}
\item {\bf INT}
\item {\bf REAL}
\item {\bf CHAR(n)}
\item {\bf REF}
\end{itemize}


The reference type, denoted by {\sc ref}, is currently
supported only for {\it in-memory} tables,
which are declared with the {\sc memory} option, as in
following example:

\begin{verbatim}
    TABLE faculty(name CHAR(20), dept CHAR(20)) MEMORY;
\end{verbatim}

In-memory tables can have columns of Reference types, which are
essentially points to tuples in some other tables (or refer to
themselves). For example, we can define the following tables:

\begin{verbatim}
    TABLE faculty(name CHAR(20), dept REF(department)) MEMORY;
    TABLE department(name CHAR(20), chair REF(faculty)) MEMORY;
\end{verbatim}

We can find out the name of the chair of the CS department by using
the following query:

\begin{verbatim}
    SELECT chair->name
    FROM department
    WHERE name = 'CS';
\end{verbatim}
\paragraph{Object ID and Path Expression}
In ATLaS, each tuple in an in-memory table has its unique OID (object
id). In SQL statements, we treat OID of a tuple as its pseudo column.
The type of the OID column is {\tt REF(table)}, where table is the table the
tuple is in.

The following example demonstrates the use of OID and reference types.

\begin{verbatim}
        TABLE tree(name char(10), father REF(tree)) MEMORY;

        INSERT INTO tree
        SELECT 'mary', t.OID
        FROM tree AS t
        WHERE t.name = 'tom';
\end{verbatim}

In the above example, we define a table with a column that refers to a
tuple in the same table. The subsequent INSERT statement creates a new
tuple whose father is another tuple in the table with name 'tom'. The
expression \verb|t.OID| retrieves the OID of the current tuple.

With reference types and OID and we can use path expressions to
navigate through the tables. The following query finds the name of
Jane's grandfather. Note that if t is of reference type, then t and
\verb|t->OID| are the same thing, which means \verb|father->name| is the same as
\verb|father->OID->name|.

\begin{verbatim}
        SELECT father->father->name
        FROM tree AS t
        WHERE t.name = 'jane';
\end{verbatim}

A critical application of reference types and path expression is the
implementation of in-memory data structures that are critical for
the performance of many algorithms, including the Apriori algorithm
discussed next.
