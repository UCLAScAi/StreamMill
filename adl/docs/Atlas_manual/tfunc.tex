
\chapter{Table Functions}
Table functions play a critical role in rearranging data, and
allowing the composition of aggregates. For instance,
if we want to count the number
of the local minima found in the execution of {\tt minpair},
we can use the following table function:


\paragraph{Cascading of Aggregates via a Tfunction}
\begin{codedisplay}
\> /* include the declaration of the {AGGREGATE} minpair here*/\\
\> TABLE stock\_prices(Day Int, Stock char(4), Cprice Real)\\
\>\>\>\>\>\>\>\>\>\>\>source  'C:$\backslash$mydabase$\backslash$stock\_prices' ;\\

\>FUNCTION localmins():(Stock Int, Point Int, Value Int)\\
\>\> \{ \> insert into RETURN\\
\>\>\>select p.Stock, minpair(p.Day, p.Cprice)\\
\>\>\>\>from stock\_prices as p\\
\>\>\>\> group by p.Stock\\
\> \>\> \} \\[0.2cm]
\>select  L.Stock,  count(L.Point)\\
\>\>from  stock\_prices, TABLE(localmins()) AS L\\
\>\>group by L.Stock; \\
\>\\
\end{codedisplay}

Here the table function does little more than calling the {\tt minpair}
aggregate on the {\tt stock\_prices} table, and organizing the results
as a virtual table with attributes  {\tt (Stock, Point, Value)}. Then,
the aggregate {\tt count} can be called on this virtual table,
whereas the direct cascading of aggregates is not allowed in
SQL, nor in ATLaS.

Also observe that the notation
{\tt TABLE( ...)} must be used when invoking table functions,
to conform to SQL standards.

The final NB is that the `dot' notation
(e.g., {\tt L.Point})  is used to refer to the various columns in a
tuple produced by a table function, whereas we use``$\rightarrow$''
for UDAs (e.g., {\tt minpair(Day, Stock) $\rightarrow$ iPoint}).

\paragraph{Pre-Sorting the Data}
The correctness of the $\tt rising$ is predicated upon the data
being sorted by increasing time. If that is not the case, we can
use a table function to pre-sort the data. Then, our program
becomes:
\begin{codedisplay}
\> /*{AGGREGATE} minpair  include here the rising declaration*/\\
\> TABLE stock\_prices(Day Int, Stock char(4), Cprice Real)\\
\>\>\>\>\>\>\>\>\>\>\>source  'C:$\backslash$mydabase$\backslash$stock\_prices' ;\\[0.1cm]
\>FUNCTION sort():(Stock Int, Point Int, Value Int)\\
\>\> \{ \> insert into RETURN\\
\>\>\>select Day, Stock, Cprice)\\
\>\>\>\>from stock\_prices\\
\>\>\>\> ORDERED BY Stock, Day\\
\> \>\> \} \\[0.1cm]
\>\>insert into stdout\\
\>\>select Stock, rising(Day, Stock) $\rightarrow$ Start,
rising(Day, Stock)$\rightarrow$ End\\
\>\>from stock\_prices, TABLE(sort())\\
\>\>group by Stock\\
\end{codedisplay}


\paragraph{Column Value Pair}The first step for most scalable classifiers
is to convert the training set into column/value pairs. For instance,
say that our training set is as follows, where the  various conditions
are described by their initials (e.g., S, O, R in the first column stand
respectively for Sunny, Overcast, and Rain):

\begin{figure}[htb]
\begin{center}
{\footnotesize
\begin{tabular}{|c|c|c|c|c|c|} \hline
{\bf RID}&{\bf Outlook}&{\bf Temp}&{\bf Humidity}&{\bf Wind}&{\bf Play}\\
\hline
1&S&H&H&W&N\\
2&S&H&L&S&N\\
3&O&H&L&W&Y\\
4&R&M&H&W&Y\\
$\ldots$& $\ldots$& $\ldots$ & $\ldots$ & $\ldots$& $\ldots$\\
\end{tabular}
}
\caption{The relation \bf PlayTennis} \label{tab:tennis}
\end{center}
\end{figure}

Then, we want to convert {\bw PlayTennis} into a new
stream of three columns {\bw (Col, Value, YorN)} by breaking down each
tuple into four records, each record representing one column in the
original tuple, including the column number, column value and the
class label {\bw YorN}. For instance, the first tuple should
produce the following
tuples:

$$\tt
(1, S, N), \ (2, H, N), \ (3, H, N), \ (4, W, N)$$
Our next table function, called
{\tt dissemble}  can be used for the task.

\begin{codedisplay}
\kw{FUNCTION} dissemble\\
\> \>(v1 \kw{Char(1)}, v2 \kw{Char(1)}, v3 \kw{Char(1)},
v4 \kw{Char(1)}, YorN \kw{Char(1)}):\\
\> \>  \> \> \> \> \> \>\hspace{2cm}  (col \kw{Int}, val \kw{Char(1)}, YorN \kw{Char(1)}); \\
\> \{\kw{INSERT} \kw{INTO} \kw{RETURN}\\
 \> \> \> \kw{VALUES} (1, v1, YorN), (2, v2, YorN),\\
  \> \> \> \> \> \>\kw{}(3, v3, YorN), (4, v4, YorN);\\
 \> \> \}\\
\end{codedisplay}

\begin{figure}[hbt]
\centering
\framebox{
\begin{tabular}{rll}
    \bnf{Tfunc-dcl} &$\rightarrow$& `FUNCTION` \bnf{id} `(` \bnf{column-list} `)` `:`
             \bnf{return-type} \bnf{Tfunc-body} \\
  \bnf{return-type} &$\rightarrow$& \bnf{type} $|$ `(` \bnf{column-list} `)`\\
  \bnf{Tfunc-body} &$\rightarrow$& `\{` \bnf{ATLaS-dcl*}  \bnf{SQL-statement*} `\}` \\

\end{tabular}
}
\caption{The declaration of a Table Function}
\end{figure}
The syntax for function tables is given below.
 The mapping expressed by {\tt dissemble} could be expressed in standard
 SQL via the union of several statements, but such a formulation could
 lead to inefficient execution. In later sections we will show that,
using this table function and specialized aggregates we can
express Naive Bayesian classifiers and Decision Tree Classifiers in
a few ATLaS statements.
