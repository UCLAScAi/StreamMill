\chapter{User-Defined Aggregates}
ATLaS supports the standard five aggregates {\tt count, sum,
avg, min}, and {\tt max} without the {\tt DISTINCT} option.
But the real power of ATLaS follows from its User-Defined Aggregates
(UDAs) discussed next.
As a first example, we define an aggregate equivalent to the standard
{\bw avg} aggregate in SQL.
\paragraph{Standard Average} The first
line of this aggregate function declares a local table, {\tt
state}, to keep the sum and count of the values processed so far.
While, for this particular example, {\bw state} contains only one
tuple, it is in fact a table that can be queried and updated using
SQL statements and can contain any number of tuples (see later
examples). These SQL statements are grouped into the three blocks
labelled respectively {\cw INITIALIZE}, {\cw ITERATE}, and {\cw
TERMINATE}. To compute the average, the SQL statement in
{\cw INITIALIZE} inserts the value taken from
the input stream and sets the count to $1$. The {\cw ITERATE}
statement updates the table by adding the new input value to the
sum and $1$ to the count. The {\cw TERMINATE} statement returns
the final result(s) of computation by {\cw INSERT INTO RETURN} (to
conform to SQL syntax, {\cw RETURN} is treated as a virtual table;
however, it is not a stored table and cannot be
used in any other role): \\

\begin{codedisplay}
\tt
\> AGGREGATE myavg(Next \kw{Int}) : Real\\
\>\{\> {TABLE} state(sum \kw{Int}, cnt \kw{Int}); \\
\>\> {INITIALIZE} : \{\\
\>\>\>INSERT INTO state {VALUES} (Next, 1);\\
\>\>\}\\
\>\>\ {ITERATE} : \{\\
\>\>\>{UPDATE} state {SET} sum=sum+Next, cnt=cnt+1;\\
%\>\>\kw{INSERT} \kw{INTO} \kw{RETURN} \\
%\>\>\>\kw{SELECT} sum/cnt \kw{FROM} state\\
%\>\>\>\kw{WHERE} cnt \% 100 = 0;\\
\>\>\}\\
\>\>{TERMINATE} : \{\\
\>\>\>{INSERT} {INTO} {RETURN} {SELECT} sum/cnt {FROM} state;\\
\>\>\}\> \\
\>\}\\
\end{codedisplay}

The basic initialize-iterate-terminate template used to define the
average aggregate of SQL-2, can now be used to defined powerful
new aggregates required by new database applications.

\paragraph{OnLine Average}For instance, there is much current interest in online
aggregates~\cite{hellerstein}. Since averages converge toward the correct
value well before all the tuples in the set have been visited, we
can have an online aggregate that returns the average-so-far
every, say, 200 input tuples. (In this way, the user or the
calling application can stop the computation as soon as
convergence is detected.) Online averages can be expressed in
ATLaS as follows:

\begin{codedisplay}
\>\kw{AGGREGATE} online\_avg(Next \kw{Int}) : \kw{Real}\\
\>\{\>\kw{TABLE} state(sum \kw{Int}, cnt \kw{Int}); \\
\>\> \kw{INITIALIZE} : \{ \\
\>\>\> \kw{INSERT INTO} state \kw{VALUES} (Next, 1);\\
\> \> \} \\
\>\> \kw{ITERATE}: \{ \\
\>\>\> \kw{UPDATE} state \kw{SET} sum=sum+Next, cnt=cnt+1;\\
\>\>\>\kw{INSERT} \kw {INTO} \kw{RETURN} \\
\>\> \> \kw{SELECT} sum/cnt \kw{FROM} state \kw{WHERE} cnt \% 200 = 0; \\
\>\>\}\\
\>\>\kw{TERMINATE} : \{  \ \}\\
\>\}\\
\end{codedisplay}

Therefore, the online average program has been obtained from the
traditional average program by removing the statements from
{\cw{TERMINATE}} and adding a {\cw{RETURN}} statement to
{\cw{ITERATE}}.  Our UDA {\bw {online\_avg}} takes a
stream of values as input and returns a stream of values as output
(one every 200 tuples). In this example only one tuple is
added to output by the  the {\small
{INSER INTO RETURN}} statement; in general, however, such statement can produce
(a stream of) several tuples. Thus ATLaS UDAs operate as
general stream transformers.

ATLaS uses the same basic framework to define both traditional
aggregates and non-blocking aggregates. ATLaS UDAs are non-blocking
when their {\cw{TERMINATE}} clause is either empty or absent.

The typical default
semantics for SQL aggregates is that the data is first sorted
according to the {\cw GROUP-BY} attributes; this is
a blocking operation.  However, ATLaS's default
semantics for UDAs is that the data is pipelined through the
{\cw INITIALIZE} and {\cw ITERATE} clauses where the input stream is transformed
into the output stream: the only blocking operations (if any) are
those specified in {\cw TERMINATE}, and only take place at the end of the
computation.


\paragraph{Calling User-Defined Aggregates (UDAs)} UDAs are called as any other builtin aggregate.  For instance,
given a database table {\tt {employee(Eno, Name, Sex, Dept,
Sal)}}, the following statement computes the average salary of
employees in department 1024 by their gender:


\begin{codedisplay}
\>\>\>\>\>\>\kw{SELECT} Sex, online\_avg(Sal)\\
\>\>\>\>\>\>\kw{FROM} employee \kw{WHERE} Dept=1024 \kw{GROUP BY}
Sex;\\
\end{codedisplay}


Thus the results of the selection, defined by {\bw Dept= 1024}, are pipelined to
the aggregate in a stream-like fashion.

\paragraph{SQLCODE}
This a convenient labor-saving device found in most
SQL systems, that comes very
handy for the ATLaS programmer who
wants to correlate a statement with
the next. SQLCODE is set to a positive value when the last
statement had a null effect, and to zero otherwise.
Thus to tell the user that no
employee was found in department $\tt 1024$, we can modify the previous
program as follows:


\begin{codedisplay}
\tt
\>\>\>\>\>\>\kw{SELECT} Sex, online\_avg(Sal)\\
\>\>\>\>\>\>\kw{FROM} employee \kw{WHERE} Dept=1024 \kw{GROUP BY}
Sex;\\
\>\>\>\>\>\>select 'Nobody found in that department'\\
\> \> \> \> \>  \> \> \> \> \>where SQLCODE $>$0;\\
\end{codedisplay}
In the last statement, the predicates in the $\tt WHERE$ clause controls its
conditional execution, in a fashion similar to that of the $\tt IF$ clauses in
a procedural programming language. In fact, the  ATLaS compiler recognizes,
and optimizes execution of, such conditional predicates.

\paragraph{Minima: Points and Values.} In the next Example, we have
a sequence of point-value pairs, and
we  define a {\bw minpair} aggregate
that returns the point where a minimum occurs
along with its value at the minimum.


\begin{codedisplay}
\kw{AGGREGATE} minpair(iPoint \kw{Int}, iValue \kw{Int}): (mPoint \kw{Int}, mValue \kw{Int})\\
 \{ \>  \kw{TABLE} mvalue(value \kw{Int}) MEMORY;
 \kw{TABLE} mpoints(point \kw{Int}) MEMORY; \\
\>\kw{INITIALIZE}: \{ \\
\>\> \kw{INSERT} \kw{INTO} mvalue \kw{VALUES} (iValue);\\
\>\> \kw{INSERT} \kw{INTO} mpoints \kw{VALUES}(iPoint);\\
\> \} \\
\>\kw{ITERATE}: \{\\
\>\> \kw{UPDATE} mvalue \kw{SET} value = iValue \kw{WHERE} iValue $<$ value;\\
\>\> \kw{DELETE FROM} mpoints \kw{WHERE} SQLCODE = 0;\\
% \>\>\kw{INSERT} \kw{INTO} mpoints \\
% \>\>\>\> \kw{SELECT} iPoint \kw{FROM} mvalue\\
% \>\>\>\> \kw{WHERE}  iValue $=$mvalue.value;\\
 \>\>\kw{INSERT} \kw{INTO} mpoints \kw{SELECT} iPoint \kw{FROM} mvalue\\
 \>\>\> \>\>\>\>\>\>\>\>\kw{WHERE}  iValue $=$mvalue.value;\\
\> \} \>\>\>\>\>\>\>\>\\
%
\>\kw{TERMINATE}: \{ \\
\>\>\kw{INSERT} \kw{INTO} \kw{RETURN} \kw{SELECT} point, value \kw{FROM} mpoints, mvalue;\\
\> \}  \\
\}\\
\end{codedisplay}

Here have used two internal tables: the {\bw mvalue} table holds, as its only entry,
the current min value, while {\bw points} holds all the points where this
value occurs.
In the {\small \kw{ITERATE}} statement we have used {\cw {SQLCODE}}
to `remember' if the previous statement updated {\bw mvalue}; this is the situation in which
the old value was larger than the new one and the old points
must be discarded.

Then, the last statement in {\cw ITERATE}
adds the new {\bw iPoint} to {\bw mpoints} if the
input value is equal to the current min value. In the UDA
definition the formal parameters of the UDA function are treated
as constants in the SQL statements. Thus, this third
{\cw INSERT} statement adds the constant value of {\bw iPoint}
to the {\bw mpoints} relation, provided that {\bw iValue} is the
same as the value
in {\bw mvalue}---thus the {\cw FROM} and {\cw WHERE} clauses
operate here as conditionals.
The {\cw RETURN} statement returns the final list of min pairs as a stream.

For instance, say that we have a time series containing the daily  closing
prices of certain stocks arranged in temporal sequence (i.e. the table {\tt stock\_prices},
below). Then the following program computes the local minima for each stock:

\begin{codedisplay}
\> /* The declaration of {AGGREGATE} minpair  should go here*/\\
\> TABLE stock\_prices(Day Int, Stock char(4), Cprice Real)\\
\hspace{6cm}source  'D:$\backslash$mydabase$\backslash$stock\_prices'\\

%\>\>insert into stdout\\
\>\>select Stock, minpair(Day, Stock) $\rightarrow$ iPoint, minpair(Day, Stock)$\rightarrow$  iValue \\
\>\>from stock\_prices\\
\>\>group by Stock  ordered by Stock, minpair(Day, Stock) $\rightarrow$ iPoint\\
\end{codedisplay}

Observe the use of  ``$\tt \rightarrow$" to identify the different
components in the two-column tuples returned by the aggregate $\tt minpair$.
Since temporal data types are not yet supported
in the current ATLaS version,
we are using integers to represent dates:
thus May, 27, 1999 is represented as {\tt 19990527}.

The next table summarizes the syntax for declaring new aggregates.

\begin{figure}[!htp]
\centering
\framebox{
\begin{tabular}{rll}
    \bnf{UDA-dcl} &$\rightarrow$& `AGGREGATE` \bnf{id} `(` \bnf{column-list} `)` `:`
             \bnf{return-type} \bnf{aggr-body}\\
    \bnf{aggr-body} &$\rightarrow$& `\{` \bnf{ATLaS-dcl*} \\
      && \ \ \ `INITIALIZE` \ `:` \{  \bnf{SQL-statement*} `;`\}` \\
      && \ \ \ `ITERATE` \ `:` \{ \bnf{SQL-statement*} `;`  `\}` \\
      && \ \ \ `TERMINATE` \ `:` \{ \bnf{SQL-statement*} `;`\}` \\
    && `\}` \\
    \bnf{return-type} &$\rightarrow$& \bnf{type} $|$ `(` \bnf{column-list} `)`\\
\end{tabular}
}
\caption{The declaration of UDAs}
\end{figure}


\paragraph{Initializing Tables and Combining Blocks.}
Let us now introduce  the following two syntactic variations of
convenience supported in ATLaS:
\begin{itemize}
\inv \item Tables declared in UDAs can be initialized as part of their declaration,
via an SQL statement. This executes at the time when the first tuple is
processed, thus the result is the same as if the initialization had
been executed in the {\small INITIALIZE} block.

\inv \item Different blocks can be merged together when they perform the
same function. In the next example the {\small INITIALIZE}
and {\small ITERATE} blocks are merged together.
\end{itemize}
Our  Online Averages UDA could also have been written as follows:
\begin{codedisplay}
\>\kw{AGGREGATE} online\_avg(Next \kw{Int}) : \kw{Real}\\
\>\{\>\kw{TABLE} state(sum \kw{Int}, cnt \kw{Int}) AS VALUES(0, 0); \\
\>\> \kw{INITIALIZE}:\kw{ITERATE}:  \{ \\
\>\>\> \kw{UPDATE} state \kw{SET} sum=sum+Next, cnt=cnt+1;\\
\>\>\>\kw{INSERT} \kw {INTO} \kw{RETURN} \\
\>\> \> \kw{SELECT} sum/cnt \kw{FROM} state \kw{WHERE} cnt \% 200 = 0; \\
\>\>\}\\
\>\}
\end{codedisplay}

In the previous example,
the  statement has been
omitted: this is equivalent to writing
 `{\cw{TERMINATE: \{ \} }}'. An empty {\kw{INITIALIZE}}
statement can also be omitted in a similar fashion.

The results produced by online averages depend
on the order in which the data is streamed through the UDA.
This illustrate a common situation in stream processing:
the  abstract semantics of the aggregate used is
order-independent, but approximations must be used
because of efficiency and real-time requirements (e.g., nonblocking
computations); often, the approximate UDA is order-dependent.

In other situations, no approximation is involved, and the
dependence on order follows from the very semantics of the UDA.
For instance, this is the case of the {\tt rising aggregate} described below.
\paragraph{Rising.} In addition to temporal extensions of standard
aggregates (suggested homework: write them in ATLaS),
TSQL2~\cite{ads}
proposes this new aggregate to return the maximal time periods during
which a certain attribute values has been increasing monotonically.
We can apply this aggregate to our
{\cdf stock\_prices(Day Int, Stock char(4), Cprice Real)} table to find
the periods during which different stocks have been rising, as follows:

\begin{codedisplay}
%\>\>insert into stdout\\
\>\>select Stock, rising(Day, Cprice) $\rightarrow$ Start,
rising(Day, Cprice)$\rightarrow$ End\\
\>\>from stock\_prices\\
\>\>group by Stock \\
\end{codedisplay}
where {\tt rising} is defined as follows:


\begin{codedisplay}
\>AGGREGATE rising(iPoint Int, iValue Real) : (Start Int, End Int)\\
\> \{ TABLE rperiod(First Int, Last Int, Value Real) MEMORY;\\
 \>\> INITIALIZE: \{ \\
\>\> \>\>  INSERT INTO rperiod VALUES (iPoint, iPoint, iValue);\\
 \>\>\> \}\\
\>\>ITERATE: \{ \\
\>\> \>\>            INSERT INTO return SELECT First, Last\\
\>\> \>\>   \>\>          FROM rperiod\\
\>\> \>\>   \>\>         WHERE  iValue $<$= Value AND First $<$ Last;\\
\>\> \>\>        UPDATE rperiod SET Last=iPoint, Value=iValue\\
\>\> \>\>   \>\>           WHERE iValue $>$  Value; \\
\>\> \>\> UPDATE rperiod SET First=iPoint, Last=iPoint, Value=iValue\\
 \>\> \>\>    \>\>              WHERE SQLCODE $>$ 0;\\
  \>\>\> \} \\

 \>\>TERMINATE:\{ INSERT INTO return SELECT First, Last \\
   \>\> \>\>   \>\>           FROM rperiod \\
     \>\> \>\>   \>\>             WHERE   First $<$ Last; \}\\
  \>\>\> \}\\
\> \}
\end{codedisplay}

Therefore we have a sequence of time-value ordered by increasing
time. We store a zero length period {\tt iPoint, iPoint} whenever
the new {\tt iValue} is not increasing (also at
{\tt INITIALIZE}). Also a non-zero length period is currently
held in {\tt rperiod} we  return it.
When the new {\tt iValue} is larger than the
previous stored value, we advance the {\tt End} of the current
period to the current point.
