\chapter{BNF Syntax of AXL}

In this section, we show the syntax of AXL in BNF. At the top-most
level, an AXL program is composed of some declarations followed by
some SQL statements. There are two types of declarations in AXL. One
is variable declaration; as of now, the only type of variable in AXL
is TABLE.  The types of statements supported in AXL are SELECT, INSERT,
UPDATE, DELETE and LOAD.


\begin{bnf}
  \begin{eqnarray*}
    <program> &->& {@ <dec>} {@ <statement> ;} \\
    <dec>  &->& <vdec> | <aggrdec>\\
    <statement> &->& <select> \\
    &&| <insert>\\
    &&| <update>\\
    &&| <delete>\\
    &&| <load>\\
  \end{eqnarray*}
\end{bnf}


\begin{bnf}
  \begin{eqnarray*}
    <vdec> &->& `TABLE` <id> `(` <columns> <keydec> `)` [<scope>] [`AS` <query>] `;`\\
    <columns> &->& <column> {@ `,`  <column>}\\
    <column> &->& <id> <type> \\
    <type> &->& `INT` | `REAL` | `CHAR` `(` <num> `)` |  `REF` `(` <id> `)`\\
    <keydec> &->& {@ `,` \ `KEY` `(` <id> {@ `,` <id>}`)` }\\
    <scope> &->& `PERSISTENT` | `LOCAL` | `MEMORY`
  \end{eqnarray*}
\end{bnf}

In addition to integer, float and string column types, we also support
a {\it reference} type. In Example~\ref{exe:reftable} , the reference
attribute, {\tt dept}, in the {\tt emp} table, refers to (i.e.,
uniquely identifies) a tuple in table {\tt dept}. Note here, the
declaration of table {\tt emp} and {\tt dept} is mutually recursive.

\begin{example}{Table Schema with Reference Types}
\begin{codedisplay}
\>\>\kw{TABLE} emp(name CHAR(20), salary INT, dept REF(dept), \kw{KEY}(name)) \kw{MEMORY};\\
\>\>\kw{TABLE} dept(name CHAR(20), manager REF(emp), \kw{KEY}(name)) \kw{MEMORY};
\end{codedisplay}
\label{exe:reftable}
\end{example}

A table in AXL can be either persistent or non-persistent (local); we
also support in-memory non-persistent tables. When a table is defined,
we can initialize its content to the result of a query by using the
{\tt AS $<$query$>$} construct. For instance, the following table
declaration initializes the table to contain two tuples.

\begin{example}{Initializing a Table}
\begin{codedisplay}
  \>\>\kw{TABLE} students(id \kw{INT}, name \kw{CHAR}(20), major \kw{CHAR}(20)) \kw{AS} \\
  \>\>\>\>\kw{VALUES}(1, 'tom', 'cs'), (2, 'jerry', 'ee');
\end{codedisplay}
\label{exe:init}
\end{example}

To intialize a table whose schema contains reference types, we need to
use the INSERT statement. In the following example, we insert a tuple
into the emp table declared in Example~\ref{exe:reftable}. Here, {\tt
  oid} is a builtin function, which takes its parameter ('sales') as
the key columns of the tuple the current tuple refers to, and returns
its OID. This part is still being implemented.

\begin{example}{Initializing a Table}
\begin{codedisplay}
\>\kw{INSERT INTO} emp {\tt VALUE} ('tom', 100, oid('sales'));
\end{codedisplay}
\label{exe:refinit}
\end{example}

The other kind of declaration is UDA. A UDA is prescribed by three
aggregate routines, INITIALIZE, ITERATE and TERMINATE, each of which
is composed of some SQL statements. If the {\tt $<$exp$>$} followed by
INITIALIZE is omitted, the INITIALIZE routine shares the code of the
ITERATE routine.  Alternatively, a UDA can be declared through
redefinition, which is being implemented.

\begin{bnf}
  \begin{eqnarray*}
    <aggrdec> &->& `AGGREGATE` <id> `(` <columns> `)` `:` <ret-type> <body>\\
    <ret-type> &->& <type> | `(` <columns> `)`\\
    <body> &->& `{`\\
      && \ \ \ {@ <vdec>}\\      
      && \ \ \ `INITIALIZE` \ `:` [{@ <statement> `;`}]\\
      && \ \ \ `ITERATE` \ `:` {@ <statement> `;`}\\
      && \ \ \ [`TERMINATE` \ `:` {@ <statement> `;`}]\\
    && `}`\\
    && | \\
      && `{`\\
      &&  \ \ \ {@ <vdec>}\\      
      &&  \ \ \ {@ <statement> `;`}\\
      && `}`
  \end{eqnarray*}
\end{bnf}

One of the features that we have added to UDAs in AXL is structured
return types. The return type of a UDA can be either a basic type or a
complex type consisting of a vector of basic types.  For example, the
following code declares a UDA which returns a value consisting of two
elements, one is of string type and the other integer type.

\begin{example}{Defining Aggregates Returning Complex Return Type}
\begin{codedisplay}
\>\>\>\kw{AGGREGATE} minsal(...): (name \kw{CHAR}(10), minsalary \kw{INT})
\end{codedisplay}
\label{exe:aggrcomplex}
\end{example}

Thus, the result of UDA {\tt minsal(..)} is a stream of 2-column
tuples.  Alternatively, we can use {\tt minsal(...)$\rightarrow$name}
and {\tt minsal(...)$\rightarrow$minsalary} to reference individual
element in the complex type. 


Following is the syntax of the query block in SQL. The query block is
the most complicated construct, and it contains a list of {\it Head
  eXPression}, a list of {\it QUaNtifiers}, an optional WHERE
condition, an optional list of GROUP BY columns, and an optional
HAVING condition.

\begin{bnf}
  \begin{eqnarray*}
    <query-block> &->& `SELECT` [`DISTINCT`] <hxp> {@ `,` <hxp>} \\
    && `FROM` <qun> {@ `,` <qun>} \\
    && [`WHERE` <exp>] \\
    && [`GROUP` \ `BY` <exp> {@ `,` <exp>}] \\
    && [`HAVING` <exp>]\\
  \end{eqnarray*}
\end{bnf}

In AXL, we support three types of quantifiers, which are i) source
tables, ii) table functions (not yet implemented), and iii)
subqueries.

\begin{bnf}
  \begin{eqnarray*}
    <qun>  &->& <id> [<qun-alias>] \\
    && | `TABLE` \ `(` <udf> `)` <qun-alias>\\
    && | `(` <query> `)` <qun-alias>
  \end{eqnarray*}
\end{bnf}

Both head expressions (hxp) and quantifiers (qun) can have aliases,
which are similar but not exactly the same.

\begin{bnf}
  \begin{eqnarray*}
    <hxp> &->& <exp> [[`AS`] <hxp-alias>]\\
      &&| [<id> `.`] `*`\\
    <hxp-alias> &->& <id> | `(` <id> {@ `,` <id>} `)`\\
    <qun-alias> &->& [`AS`] <id> [`(` <id> {@ `,` <id>} `)`]\\
    <udf> &->& <id> `(` [<exp> {@ `,` <exp>}] `)`\\
  \end{eqnarray*}
\end{bnf}

Since AXL supports UDA returning complex data types, we make it easier
to define aliases for the complex data types returned by UDAs.  We can
write:

\begin{codedisplay}
\>\>\>\kw{SELECT} minsal(...) \kw{AS} (name, salary) \\
\>\>\>\kw{FROM} t
\end{codedisplay}

where the aggregate {\tt minsal} is declared in
Example~\ref{exe:aggrcomplex}, as a shorthand of:

\begin{codedisplay}
\>\>\>\kw{SELECT}\>\>\>\>minsal(...)$\rightarrow$ name \kw{AS} name,\\
\>\>\>\>\>\>\>minsal(...)$\rightarrow$ minsalary \kw{AS} salary\\
\>\>\>\kw{FROM} t
\end{codedisplay}

We connect query blocks using UNION, INTERSECT and EXCEPT set
operators to form a query. (The INTERSECT and EXCEPT operators are not
yet implemented.) The SQL SELECT statement is composed of a query and
an optional ORDER BY clause.

\begin{bnf}
  \begin{eqnarray*}
    <select> &->&  <query> [<order-clause>]\\
    <order-clause> &->& `ORDER` \ `BY` <exp> [`ASC`|`DSC`] {@ `,` <exp> [`ASC`|`DSC`] }\\
    <query>  &->& <query> (`UNION`|`INTERSECT`|`EXCEPT`) [`ALL`] <query-block> \\
  \end{eqnarray*}
\end{bnf}

The support of CREATE, DELETE, INSERT and UPDATE statements conforms
with the SQL standard. We also support subqueries and table functions
in AXL. We also have a LOAD statement, which bulk loads text data into
AXL system.

\begin{bnf}
  \begin{eqnarray*}
    <delete> &->& `DELETE` \ `FROM` <id> [`WHERE` <exp>]\\
    <insert> &->& `INSERT` \ `INTO` <id> <query>\\
    <update> &->& `UPDATE` <id> `SET` <updates> [`WHERE` <exp>]\\
    <updates> &->& <id> `=` <exp> {@ `,` <id> `=` <exp>}\\
    <load> &->& `LOAD` \ `FROM` <id> `INTO` <id>\\
  \end{eqnarray*}
\end{bnf}


\begin{bnf}
  \begin{eqnarray*}
    <exp> &->& `NIL`\\
    && | <num>\\
    && | <float>\\
    && | <string>\\
    && | <id> [ `.` <id>] \\
    && | <ref>\\
%    && | <id> `(` {@ <exp> {@, <exp>} } `)`\\
%   && | `(` {@ <exp> `;`} `)`\\
    && | <exp> (`+`|`-`|`*`|`/`|`%`) <exp>\\
    && | <exp> (`=`|`!=`|`<`|`<=`|`>`|`>=`) <exp>\\
    && | <exp> (`AND`|`OR`|`IN`|`NOT`\ `IN`) <exp>\\
    && | `EXISTS` <exp>\\
    && | (`max`|`min`|`count`|`sum`|`avg`) `(`<exp>`)`\\
    && | <udf> [ `->` <id>]\\
    && | <case-exp>\\
    && | `(` <exp> `)`\\
    && | `(` <query> `)`\\
    && | `{` {@ <vdec>} {@ <exp> `;`} `}` \\
    <ref> &->& <ref> `->` <id>\\
    && | <id> [`.` <id>] `->` <id>
  \end{eqnarray*}
\end{bnf}

As shown in the above syntax, we support path expression in AXL. For
example, in the following query, we find the name of the department
for each employee.

\begin{example}{Path Expression}
\begin{codedisplay}
\>\kw{SELECT} name, dept$\rightarrow$name \\
\>\kw{FROM} emp
\end{codedisplay}
\label{exe:refferencing}
\end{example}


AXL supports CASE expressions. A CASE expression can be used wherever
an expression such as {\tt x+y} or {\tt foo(x)} can be used. CASE can
be used in a SELECT clause, in a WHERE clause, or in the SET clause of
an UPDATE statement.

\begin{bnf}
  \begin{eqnarray*}
    <case-exp> &->& `CASE` <exp> <when-exp-list> [`ELSE` <exp>] `END`\\
    && | `CASE` <when-exp-list>  [`ELSE` <exp>] `END`\\
    <when-exp-list> &->& `WHEN` <exp> `THEN` <exp> {@ `WHEN` <exp> `THEN` <exp>}\\
    <id> &->& <letter> {@ <letter> | <digit>}\\
    <letter> &->& (`a`-`z`|`A`-`Z`)\\
    <digit> &->& (`0`-`9`)\\
  \end{eqnarray*}
\end{bnf}

In its simple form, a CASE expression evaluates
to one of several result expressions, depending on the value of a test
expression. The CASE expression also has a more general form, which
consists of a set of search conditions, each paired with a result
expression.

