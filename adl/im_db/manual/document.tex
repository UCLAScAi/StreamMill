\documentclass[11pt]{article}
\renewcommand{\baselinestretch}{1.2}

%revision by CZ---abstract not done
%\usepackage{latex8}
%\usepackage{times}
\usepackage{latexsym}
\usepackage{graphicx}
\usepackage{algorithm}
\usepackage{algorithmic}
%\pagestyle{empty}

%-------------------- HYPHENATIONS ------------------------
\hyphenation{PAGED-ARRAY LINKED-LIST}

%-------------------- DEFINITIONS ------------------------
\newcommand{\phsp}{\hspace*{1cm}}
\newcommand{\nhsp}{\hspace*{-1cm}}
\newcommand{\inv}{\vspace*{-0.2cm}}

\newcommand{\LDL}{\mbox{$\cal LDL$}}
\newcommand{\gl}{\gamma_{L}}
%---------------------------------------------------------
\newcommand{\bldl}{\smallskip\[\begin{array}{ll}\small}
\newcommand{\cldl}{\vspace*{-0.5cm}\[\begin{array}{ll}\small}
\newcommand{\eldl}{\end{array}\]\rm}
\newcommand{\prule}[2]{\tt #1 \leftarrow & \tt #2 \\}
\newcommand{\lrule}[2]{\tt #1 \leftarrow & \tt #2 \ \ \ \ \ \ \Box\\}
\newcommand{\pfact}[2]{\tt #1 & \tt #2 \\}
\newcommand{\pbody}[2]{\tt #1 & \tt #2 \\}
\newcommand{\lbody}[2]{\tt #1 & \tt #2 \ \ \ \ \ \Box\\}
%---------------------------------------------------------

%%%%%%%%%%%%%%%%%%%from KDD%%%%%%%%%%%%%%%%%%%%%%%%%%%%
\newtheorem{theorem}{Theorem}
\newtheorem{lemma}[theorem]{Lemma}
\newtheorem{claim}[theorem]{Claim}
\newtheorem{proposition}[theorem]{Proposition}
\newtheorem{corollary}[theorem]{Corollary}
\newtheorem{example}{Example}
\newtheorem{definition}{Definition}

\def\newdef#1{\emph{#1}}
%SQL displays
%\def\cdf{\sffamily}
\def\cdf{\sf }
\def\cdf{\sf }
%%%%%%%%%%%%%%%%%%%%%%%%%%%
\newcommand{\AXL}{$\cal AXL$}
\newcommand{\IMDB}{{\small{\cdf IM\_DB}}}
\newcommand{\IMREL}{{\small{\cdf IM\_REL}}}
\newcommand{\IMRELcursor}{{\small{\cdf IM\_REL\_cursor}}}
\newcommand{\DBNOTFOUND}{{\small{\cdf DB\_NOTFOUND}}}
\newcommand{\DBKEYEMPTY}{{\small{\cdf DB\_KEYEMPTY}}}
\newcommand{\DBGETBOTH}{{\small{\cdf DB\_GET\_BOTH}}}
\newcommand{\DBSETRECNO}{{\small{\cdf DB\_SET\_RECNO}}}
\newcommand{\DBGETOIDOFKEY}{{\small{\cdf DB\_GET\_OIDOFKEY}}}
\newcommand{\DBGETOID}{{\small{\cdf DB\_GET\_OID}}}
\newcommand{\DBFIRST}{{\small{\cdf DB\_FIRST}}}
\newcommand{\DBNEXT}{{\small{\cdf DB\_NEXT}}}
\newcommand{\DBLAST}{{\small{\cdf DB\_LAST}}}
\newcommand{\DBPREV}{{\small{\cdf DB\_PREV}}}
\newcommand{\DBCURRENT}{{\small{\cdf DB\_CURRENT}}}
\newcommand{\DBNEXTDUP}{{\small{\cdf DB\_NEXT\_DUP}}}
\newcommand{\DBSET}{{\small{\cdf DB\_SET}}}
\newcommand{\DBAPPEND}{{\small{\cdf DB\_APPEND}}}
\newcommand{\dbrecnot}{{\small{\cdf db\_recno\_t}}}

%%%%%%%%%%%%%%%%%%%%%%%%%%%
\def\brack#1{\langle{#1}\rangle}
\def\newdef#1{\emph{#1}}
\def\codefont{\sffamily} %Note: < and > are not in this font; use $<$ and $>$
\def\inv{\vspace*{-0.16cm}}
\def\smb{\hspace*{-0.16cm}}
\def\kw#1{\bf{\cdf #1}}
\def\mysize#1{\normalsize{#1}}
\newenvironment{codedisplay}
{\renewcommand{\baselinestretch}{1}\vspace*{-\partopsep}\cdf\addtolength{\baselineskip}{-1pt}
\samepage\begin{tabbing}
\quad
%\ \ \ \ \ \ \ \ \=\ \ \ \ \ \ \ \ \=\ \ \ \ \ \ \ \ \=\ \ \ \ \ \ \ \ \=
%\ \ \ \ \ \ \ \ \=\ \ \ \ \ \ \ \ \=\ \ \ \ \ \ \ \ \=\ \ \ \ \ \ \ \ \=
\ \ \=\ \ \ \ \=\ \ \ \ \=\ \ \ \ \=
\ \ \=\ \ \ \ \=\ \ \ \ \=\ \ \ \ \=
\ \ \=\ \ \ \ \=\ \ \ \ \=\ \ \ \ \=
\kill}{\end{tabbing}\vspace*{-\partopsep}%\vspace*{-\topsep}\vspace*{-\parsep}
}

\newenvironment{codeexample}[1]{\begin{example}{#1}\begin{codedisplay}}{\end{codedisplay}\end{example}}

%\markright{Paper id 249}
\pagestyle{plain}

\addtolength{\topmargin}{-0in}
\addtolength{\textheight}{+0.95in}
\addtolength{\oddsidemargin}{-0.5in}
\addtolength{\textwidth}{+1in}

%%%%%%%%%%%%%%%%%%%%%%%%%%%%%%
\begin{document}

%\thispagestyle{empty}

%\newpage
\setcounter{page}{1}

\begin{center}
{\LARGE \bf In-Memory Database Design and API}\\
Version 0.5\\
\vspace{0.25cm}
{\large {\bf Jiejun Kong}, {\bf Haixun Wang} and {\bf Carlo Zaniolo}\\
Computer Science Department\\ 
University of California at Los Angeles\\ 
Los Angeles, CA 90095}\\
\small{ \{jkong,hxwang,zaniolo\}@cs.ucla.edu}
%\maketitle
\end{center}


%==================================================

%\begin{abstract}
%\end{abstract}

%\tableofcontents

%==================================================
\section{Design}

The in-memory database for {\AXL} is designed to implement,
\begin{enumerate}
\item Sequential scan over all tuples;
\item Potential hash index which can be either unique or non-unique;
\item Potential variable length tuples (in next versions).
\end{enumerate}

We consider two solutions:
\begin{enumerate}
\item Obviously a simple linked list can do the job.

      Each item of the list is a tuple plus some control information
      (object id and status flags) associated with it.

      Sequential scan is a linked list scan.  Variable length tuples
      are accomodated by malloc() variable length space on heap.

      Hash index can be built on the linked list.  If hash index is
      unique, it is trivial to deal with it.  Otherwise, an extra
      pointer is defined in control information fields of each tuple
      so that tuples having same hash key can be chained together.

      Right now we allow only one non-unique hash index on an
      in-memory relation because only one extra hash-chain pointer is
      defined in control information of each tuple.
\item The above solution may suffer from too many malloc() calls since
      each tuple insertion will issue a malloc() call.

      This can be solved by paging.  Each time we malloc a page (fixed
      size wrt. each relation) rather than a tuple.  As each page
      hosts multiple tuples, cost of malloc() is decreased by blocking
      factor (number of tuples in a page).

      However, this comes with tradeoffs---although cost of malloc()
      is decreased, inter-page and intra-page control information must
      be maintained with certain time and space cost.

      Inter-page control information includes connection links among
      different pages so that a sequential scan is possible.

      Intra-page control information includes (a) a list of all in-use
      slices, (b) a list of all free slices in a page.
\end{enumerate}

Since performance doesn't vary much between a simple linked list
implementation and a paged linked list implementation.  Currently I
have only implemented a simple linked list implementation.

\section{Implementation}
\begin{itemize}
\item {\tt im\_rel.c} implements in-memory relations as linked lists.
\item {\tt im\_cursor.c} implements cursor operations on in-memory
      relations.
\item {\tt im\_debug.c} has some debugging routines and sample calls.
\item {\tt hash-arbitrary.c} is adapted from GNU Assembler software
      package.  The original version can only handle text strings
      because it was designed for looking up symbol table.  I change
      the code to handle arbitrary bit strings by passing a ``void *''
      and its length in bytes.
\item {\tt im\_db\_env\_method.c} {\tt im\_db\_err.c} emulate BerkeleyDB's
      error reporting utilities.
\item {\tt mylib.c} has some personal library functions.
\end{itemize}

In the linked-list implementation I use a doubly-linked cyclic list
because we need to support bi-directional cursor operations.

Another link pointer is added for chaining duplicate keys.  This is
because we need to support {\DBNEXTDUP} operation during a cursor
scan.

To support object-relational database features, the in-memory database
associates an OID with each tuple.  OID is implemented as memory
address of a tuple.

OID also affects deletions and retrievals.  Tuples in an OID relation
are never deleted because it is too expensive to update all places
referencing the deleting tuples.  Instead, we set a {\em tombstone}
flag on the deleting tuples, thus each tuple also has a flag field.

A tombstoned tuple keeps all link pointers.  However, during a get
operation it is skipped.  When a tombstoned tuple is referenced by a
path-expression instance, it is a signal indicating that no result is
returned for the path-expression instance.
\newpage
%==================================================
\section{API: Application Programming Interface}

\subsection{im\_rel\_create}
\begin{verbatim}
#include <im_db.h>
int im_rel_create(IM_REL **relpp, IM_DB_ENV *dbenv, IM_DBTYPE type, u_int32_t flags);
\end{verbatim}
\begin{description}
\item[Description]\ \\
  The {\em im\_rel\_create} function creates handle for an in-memory
  relation.  A pointer to this structure is returned in the memory
  referenced by AXL.

  If the dbenv argument is NULL, the database is standalone, i.e., it
  is not part of any In-memory DB environment.

  If the dbenv argument is not NULL, the database is created within
  the specified environment. The database access methods automatically
  make calls to the other subsystems. Right now only error reporting
  is considered in the environment.  Feel free to use NULL here.

  The currently supported in-memory DB file formats (or access methods)
  is IM\_LINKED\-LIST.  Other formats (e.g. IM\_ARRAY) may be
  supported upon request.
  
  The IM\_LINKEDLIST format is virtually a linked list.  The order of
  tuples in a linked list follows the order they are inserted.
  
  The flags value must be set to 0 or by bitwise inclusively OR'ing
  together one or more of the following values. 
  \begin{description}
  \item[IM\_REL\_INDEXED]\ \\
       A hash-index should be built on the in-memory relation.\\
       If such a hash-index is non-unique, a hash-chain is maintained
       in the relation so that tuples having same hash key are chained
       together.
  \item[IM\_OID]\ \\
       When relation $A$ has an attribute which is a foreign key of
       relation $B$, normally relations will use a value stored in $A$
       as a key of $B$ and locate qualified tuples in $B$.\\
       When IM\_OID is specified, tuples in $B$ will be referenced via
       pointers (so-called ``object referenced'') rather than values.
       Therefore, a tuple in $B$ can {\em not} be deleted. Otherwise,
       pointers stored in $A$ will point to wrong place.  Instead, a
       tombstone flag is set for all deleted tuples in $B$.\\
       To let relation $B$ behave in this way, one must specify
       IM\_OID when $B$ is created.  Currently OID is implemented as
       memory address in in-memory relations.
  \end{description}
\item[Caveats]\ \\
  Right now only fixed-length tuples are considered.  Otherwise, when
  a variable-length tuple is updated, the new tuple may exceed the
  old tuple in length of bytes, we must allocate a new area for the
  longer new tuple and set an extra pointer in the old tuple pointing
  to the new tuple.
\item[Return Value]\ \\
  The {\em im\_rel\_create} function returns a non-zero error value on
  failure and 0 on success.
\item[Errors]\ \\
  The {\em im\_rel\_create} function may fail and return a non-zero
  error for errors specified for other {\IMDB} and C library or system
  functions.
\item[Sample Call]\ 
\begin{verbatim}
if((ret = im_rel_create(&relp, NULL, IM_LINKEDLIST,
                        IM_REL_INDEXED | IM_OID)) != 0)
    return ERROR;
\end{verbatim}
\end{description}

\newpage
\subsection{{\IMREL}-$>$open}
\begin{verbatim}
#include <im_db.h>
int IM_REL->open(IM_REL *relp, char *name, u_int32_t flags);
\end{verbatim}
\begin{description}
\item[Description]\ \\
  The {\IMREL}-$>$open interface opens the database represented by name for
  both reading and writing. The argument `name' is for Berkeley DB
  compatibility.

  The flags parameter must be set to 0 currently.  

\item[Return Value]\ \\
  The {\IMREL}-$>$open function returns a non-zero error value on failure
  and 0 on success.
\item[Errors]\ \\
  The {\IMREL}-$>$open function may fail and return a non-zero
  error for errors specified for other {\IMDB} and C library or
  system functions.
\end{description}

\newpage
\subsection{{\IMREL}-$>$close}
\begin{verbatim}
#include <im_db.h>
int IM_REL->close(IM_REL *rel, u_int32_t flags);
\end{verbatim}
\begin{description}
\item[Description]\ \\
The {\IMREL}-$>$close function closes any open cursors, frees any
allocated resources.

The flags parameter must be set to 0 currently.  

Once {\IMREL}-$>$close has been called, regardless of its return, the
{\IMREL} handle may not be accessed again until the handle is
re-opened.
\item[Return Value]\ \\
The {\IMREL}-$>$close function returns a non-zero error value on failure
and 0 on success.
\item[Errors]\ \\
The {\IMREL}-$>$close function may fail and return a non-zero error for
errors specified for other {\IMDB} and C library or system functions.
\end{description}

\newpage
\subsection{{\IMREL}-$>$del}
\begin{verbatim}
#include <im_db.h>
int IM_REL->del(IM_REL *rel, DBT *key, u_int32_t flags);
\end{verbatim}
\begin{description}
\item[Description]\ \\
  The {\IMREL}-$>$del function removes key/data pairs from a
  relation. The key/data pair associated with the specified key is
  discarded from the relation. In the presence of duplicate key
  values, all records associated with the designated key will be
  discarded.

  The flags parameter is currently unused, and must be set to 0.
\item[Return Value]\ \\
  The {\IMREL}-$>$del function returns a non-zero error value on
  failure, 0 on success, and returns {\DBNOTFOUND} if the specified key
  did not exist in the file.
\item[Errors]\ \\
The {\IMREL}-$>$del function may fail and return a non-zero error for
errors specified for other {\IMDB} and C library or system functions.
\end{description}

\newpage
\subsection{{\IMREL}-$>$get}
\begin{verbatim}
#include <im_db.h>
int IM_REL->get(IM_REL *rel, DBT *key, DBT *data, u_int32_t flags);
\end{verbatim}
\begin{description}
\item[Description]\ \\
  The {\IMREL}-$>$get function retrieves key/data pairs from the database. The
  address and length of the data associated with the specified key are
  returned in the structure referenced by data.

  In the presence of duplicate key values, {\IMREL}-$>$get will return the
  first data item for the designated key. Duplicates are reversely
  sorted by storing order (Inserting at head of a hash-chain is
  faster, thus tuples are reversely sorted by storing order).
  Retrieval of duplicates requires the use of cursor operations. See
  {\IMRELcursor}-$>$c\_get for details.

  The flags parameter must be set to 0 or one of the following values: 
  \begin{description}
  \item[{\DBGETBOTH}]\ \\
    Retrieve the key/data pair only if both the key and data match the
    arguments.
  \item[{\DBSETRECNO}]\ \\
    Retrieve the specified numbered key/data pair from a
    database. Upon return, both the key and data items will have been
    filled in, not just the data item as is done for all other uses of
    the {\IMREL}-$>$get function.

    The `data' field of the specified key must be a pointer to a
    logical record number (i.e., a {\dbrecnot}). This record number
    determines the record to be retrieved.

    The `size' field of the specified key must indicate the length of
    the key field instead of {\dbrecnot}.
  \item[{\DBGETOID}]\ \\
    Given an OID, retrieve the key/data pair from a database.  The OID
    is passed in `key' argument.  Upon return, both the key and data
    items will have been filled in, as {\DBSETRECNO} does.
  \item[{\DBGETOIDOFKEY}]\ \\
    Retrieve OID of a given key.  If there are duplications in key
    values, the OID of the first one is returned.  The returned OID is
    place in `data' argument.
  \end{description}
  
\item[Return Value]\ \\
  Returned data is placed in a system-allocated buffer, just as what
  BerkeleyDB does.  Users only need to take care of input arguments.

  If the requested key is not in the database, the {\IMREL}-$>$get
  function returns {\DBNOTFOUND}.

  Otherwise, the {\IMREL}-$>$get function returns a non-zero error
  value on failure and 0 on success. 

\item[Errors]\ \\
  The {\IMREL}-$>$get function may fail and return a non-zero error
  for errors specified for other Berkeley DB and C library or system
  functions.

\item[Sample Calls]\ 
\begin{verbatim}
DBT key, data;
db_recno_t i;

/* demo of DB_SET_RECNO */
i = 10;   /* retrieve 10th record */
key.data = &i, key.size = sizeof(db_recno_t);
if((ret = im_rel->get(relp, &key, &data, DB_SET_RECNO)) != 0)
    return ERROR;

/* demo of DB_GET_OIDOFKEY / DB_GET_OID */
key.data = "a-key", key.size = sizeof("a-key");
if((ret = im_rel->get(relp, &key, &data, DB_GET_OIDOFKEY)) != 0)
    return ERROR;
key.data = data.data, key.size = data.size;
if((ret = im_rel->get(relp, &key, &data, DB_GET_OID)) != 0)
    return ERROR;
\end{verbatim}
\end{description}

\newpage
\subsection{{\IMREL}-$>$put}
\begin{verbatim}
#include <im_db.h>
int IM_REL->put(IM_REL *rel, DBT *key, DBT *data, u_int32_t flags);
\end{verbatim}
\begin{description}
\item[Description]\ \\
  The {\IMREL}-$>$put function stores key/data pairs in the relation. 

  The flags parameter must be set to 0 or one of the following values: 
  \begin{description}
  \item[{\DBAPPEND}]\ \\
    Append a tuple to the existing linked list, or if the linked list
    doesn't exist, treat the new tuple as the head.  If a hash index
    has been created on the table, the hash-chain should be
    maintained.  Currently the newly inserted tuple is inserted at the
    end of a hash-chain.
  \end{description}

  The default behavior of the {\IMREL}-$>$put function is to enter the
  new key/data pair, replacing the first previously existing key.
\item[Return Value]\ \\
  The {\IMREL}-$>$put function returns a non-zero error value on
  failure, 0 on success.
\item[Errors]\ \\
  The {\IMREL}-$>$put function may fail and return a non-zero error
  for errors specified for other {\IMDB} and C library or system
  functions.
\item[Caveat]\ \\
  BerkeleyDB users please follow the following policy to make {\IMDB}
  and BerkeleyDB compatible:
  \begin{itemize}
  \item If DUP flag is allowed by DB-$>$set\_flags() in BerkeleyDB,
        then always call {\IMREL}-$>$put() with {\DBAPPEND} flag in
        the {\IMDB} version.
  \item If DUP flag is disallowed by DB-$>$set\_flags() in BerkeleyDB,
        then always call {\IMREL}-$>$put() with flag 0 in
        the {\IMDB} version.
  \end{itemize}
\end{description}

\newpage
\subsection{{\IMREL}-$>$cursor}
\begin{verbatim}
#include <im_db.h>
int IM_REL->cursor(IM_REL *relp, IM_RELC **cursorp, u_int32_t flags);
\end{verbatim}
\begin{description}
\item[Description]\ \\

  The IM\_REL-$>$cursor function creates a cursor and copies a pointer
  to it into the memory referenced by ``cursorp''.

  A cursor is a structure used to provide sequential access through a
  database.
  
  Currently the flags value must be set to 0.

\item[Return Value]\ \\
  The IM\_REL-$>$cursor function returns a non-zero error value on
  failure and 0 on success.
\end{description}

\newpage
\subsection{{\IMRELcursor}-$>$c\_get}
\begin{verbatim}
#include <im_db.h>
int IM_REL_cursor->c_get(IM_RELC *cursor, DBT *key, DBT *data, u_int32_t flags);
\end{verbatim}
\begin{description}
\item[Description]\ \\

  The flags parameter must be set to one of the following values: 
  \begin{description}
  \item[{\DBFIRST}]\ \\
    The cursor is set to reference the first tuple of the relation,
    and that pair is returned.

    If the database is empty, {\IMRELcursor}-$>$c\_get will return
    {\DBNOTFOUND}.
  \item[{\DBNEXT}]\ \\
    If the cursor is not yet initialized, {\DBNEXT} is identical to
    {\DBFIRST}.

    Otherwise, the cursor is moved to the next tuple of the database,
    and that pair is returned.

    If the cursor is already on the last record in the database,
    {\IMRELcursor}-$>$c\_get will return {\DBNOTFOUND}.
  \item[{\DBLAST}]\ \\
    The cursor is set to reference the last tuple of the relation,
    and that pair is returned.

    If the database is empty, {\IMRELcursor}-$>$c\_get will return
    {\DBNOTFOUND}.
  \item[{\DBPREV}]\ \\
    If the cursor is not yet initialized, {\DBPREV} is identical to
    {\DBLAST}.

    Otherwise, the cursor is moved to the previous tuple of the database,
    and that pair is returned.

    If the cursor is already on the first record in the database,
    {\IMRELcursor}-$>$c\_get will return {\DBNOTFOUND}.
  \item[{\DBCURRENT}]\ \\
    Return the tuple currently referenced by the cursor. 

    If the cursor is not yet initialized, the {\IMRELcursor}-$>$c\_get function
    will return EINVAL.
  \item[{\DBSET}]\ \\
    Move the cursor to the specified key/data pair of the database,
    and return the datum associated with the given key. 

    In the presence of duplicate key values, {\IMRELcursor}-$>$c\_get
    will return the first data item for the given key. 

    If no matching keys are found, {\IMRELcursor}-$>$c\_get will return
    {\DBNOTFOUND}.
  \item[{\DBGETBOTH}]\ \\
    The {\DBGETBOTH} flag is identical to the {\DBSET} flag, except
    that both the key and the data arguments must be matched by the
    key and data item in the database.
  \item[{\DBNEXTDUP}]\ \\
    Return the next tuple in the hash-chain, i.e., next tuple having
    same hash key as current one, when hash index exists.  Hash-chain
    is not maintained if hash index doesn't exist.
    {\IMRELcursor}-$>$c\_get will return {\DBNOTFOUND} if hash-chain
    reaches its end.

    If the cursor is not yet initialized, the {\IMRELcursor}-$>$c\_get
    function will return EINVAL.

  \item[{\DBGETOIDOFKEY}]\ \\
    Retrieve OID of the tuple currently referenced by the cursor.  The
    returned OID is place in `data' argument.
  \end{description}

  `Tombstoned' tuples are considered deleted and will be ignored
  during cursor searching.
\item[Return Value]\ \\
   If no matching keys are found, {\IMRELcursor}-$>$c\_get will return
   {\DBNOTFOUND}.  Otherwise, the {\IMRELcursor}-$>$c\_get function
   returns a non-zero error value on failure and 0 on success.

   If {\IMRELcursor}-$>$c\_get fails for any reason, the state of the
   cursor will be unchanged.

\item[Error]\ \\ 
The {\IMRELcursor}-$>$c\_get function may fail and return a non-zero
error for errors specified for other {\IMDB} and C library or system
functions.

\item[Sample Call]\ 
\begin{verbatim}
DBT key, data;

if((ret = im_rel_cursor->c_get(relcp, &key, &data, DB_FIRST)) != 0)
    return ERROR;
\end{verbatim}
\end{description}

\newpage
\subsection{{\IMRELcursor}-$>$c\_put}
\begin{verbatim}
#include <im_db.h>
int IM_RELcursor->c_put(IM_RELC *cursorp, DBT *key, DBT *data, u_int32_t flags);
\end{verbatim}
\begin{description}
\item[Description]\ \\

  The IM\_RELcursor-$>$c\_put function stores key/data pairs into the
  database.

  The flags parameter must be set to one of the following values: 
  \begin{description}
  \item[{\DBCURRENT}]\ \\
    Overwrite the data of the key/data pair referenced by the cursor
    with the specified data item. The input parameters are ignored. 
  \end{description}

\item[Return Value]\ \\
  If the cursor is not yet initialized, the IM\_RELcursor-$>$c\_put
  function will return {\DBNOTFOUND}. Otherwise, the
  IM\_RELcursor-$>$c\_put function returns 0 on success. 
\end{description}

\newpage
\subsection{{\IMRELcursor}-$>$c\_del}
\begin{verbatim}
#include <im_db.h>
int IM_RELcursor->c_del(IM_RELC *cursorp, u_int32_t flags);
\end{verbatim}
\begin{description}
\item[Description]\ \\

  The IM\_RELcursor-$>$c\_del function deletes the key/data pair currently
  referenced by the cursor. 

  The flags parameter is currently unused, and must be set to 0. 

  To be compatible with BerkeleyDB's behavior (according to Haixun and
  James's code),  the cursor position is automatically located at
  previous available data item after a delete, or set as
  un-initialized if the head is being deleted.  This design will make
  {\IMRELcursor}-$>$c_get always work with {\DBNEXT} flag on.

\item[Return Value]\ \\
  If the cursor is not yet initialized, the IM\_RELcursor-$>$c\_del
  function will return {\DBNOTFOUND}. Otherwise, the
  IM\_RELcursor-$>$c\_del function returns 0 on success. 
\end{description}

\newpage
\subsection{{\IMRELcursor}-$>$c\_close}
\begin{verbatim}
#include <im_db.h>
int IM_RELcursor->c_close(IM_RELC *cursorp);
\end{verbatim}
\begin{description}
\item[Description]\ \\

  The IM\_RELcursor-$>$c\_close function discards the cursor. 

  Once IM\_RELcursor-$>$c\_close has been called, regardless of its
  return, the cursor handle may not be used again. 

\item[Return Value]\ \\
  The IM\_RELcursor->c\_close function returns a non-zero error value
  on failure and 0 on success.
\end{description}

\newpage
\tableofcontents
\end{document}
